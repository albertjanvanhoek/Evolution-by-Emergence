\documentclass[11pt,a4paper]{article}

% ---------- Encoding & language ----------
\usepackage[T1]{fontenc}
\usepackage[utf8]{inputenc}
\usepackage[english]{babel}

% ---------- Typography ----------
\usepackage{lmodern}
\usepackage{microtype}

% ---------- Layout ----------
\usepackage{geometry}
\geometry{margin=1in}

% ---------- Math & symbols ----------
\usepackage{amsmath, amssymb}

% ---------- Lists ----------
\usepackage{enumitem}
\setlist{nosep}

% ---------- Hyperlinks ----------
\usepackage{hyperref}
\hypersetup{
    colorlinks=true,
    linkcolor=black,
    urlcolor=blue,
    pdftitle={Freedom Under Forced Free Will},
    pdfauthor={Albert Jan van Hoek},
}

% ---------- Title ----------
\title{\textbf{Freedom Under Forced Free Will}}
\author{Albert Jan van Hoek}
\date{}

\begin{document}

\maketitle

\section*{Introduction}

Freedom is often defined negatively or expressively: \emph{``I can do what I want''}, \emph{``I can say what I think''}, or \emph{``I can choose without interference''}. While these intuitions capture something important, they obscure a deeper structural reality. In complex, far-from-equilibrium systems, agents are never unconstrained. Existence itself imposes requirements: bodies must be maintained, resources accessed, environments preserved. Under these conditions, freedom cannot mean the absence of obligation. Obligation is unavoidable.

Within the framework of forced free will, freedom is therefore not the ability to escape necessity, but the quality of one’s relationship to it.

\section*{Intelligence as a Network Phenomenon}

The foundational principle of this framework is simple but non-negotiable: \emph{intelligence is a network phenomenon}. It is dependent on its substrate—the underlying biological, social, and ecological networks—and to exist over time it must sustain the integrity of those networks. This is not an ethical claim but a structural one. From this single principle, the remainder of the framework follows.

If intelligence is network-dependent and networks must be maintained to persist, then several common intuitions collapse. Individual intelligence is not a self-contained property but a useful local fiction masking distributed process. ``Rational self-interest,'' when untethered from network health, becomes structurally irrational—ultimately self-terminating. The commons is not a moral ideal layered on top of intelligence; it is the substrate of cognition itself. And freedom cannot be non-relational, because the entity doing the choosing is itself a relational phenomenon.

This reframes the intelligence discourse. We tend to ask how intelligent a person or system is, as if intelligence were something possessed. But this is a category error. Asking how intelligent an individual is is like asking how wet a single water molecule is: the molecule participates in wetness; it does not contain it. Likewise, individuals participate in intelligence as enacted by networks.

From this perspective, sustainability is no longer optional or value-contingent. The imperative to preserve environmental, social, and informational substrates does not arise because it is ethically desirable, but because intelligence cannot survive substrate collapse. Any rule, structure, or limitation can therefore be evaluated by a single criterion: does it sustain the integrity of the networks that enable intelligence to persist?

\section*{Freedom as Access to Choice}

Under forced free will, freedom is best understood as \emph{access to meaningful choice}. This access depends on two conditions:
\begin{enumerate}
    \item Diversity of available options
    \item Capacity to select among them
\end{enumerate}

An agent is not free simply because options exist in principle. Options must be reachable, legible, and actionable. A society with many theoretical possibilities but high barriers to entry is less free than one with fewer but widely accessible paths. Likewise, an agent with formal autonomy but without the means to act is not meaningfully free.

Freedom, in this sense, is not about unlimited possibility, but about structured opportunity.

\section*{Forced Action, Expanded Freedom}

Forced free will begins with a hard claim: action is mandatory. To exist is to act; even non-action is a choice with consequences. However, this does not negate freedom. Instead, it relocates it.

If required actions are:
\begin{itemize}
    \item clear rather than opaque,
    \item easy rather than costly,
    \item accessible rather than exclusive,
\end{itemize}
then the agent’s remaining degrees of freedom expand. When survival tasks are simplified, shared, or infrastructurally supported, cognitive and temporal capacity is released. This surplus becomes freedom: the ability to explore, cooperate, experiment, and recombine.

This is why infrastructure matters. Public transport, for example, is not merely a convenience; it is freedom-enhancing because it lowers the cost of movement, increases reachable options, and decouples participation from private ownership. The same logic applies across the commons: education, information, energy, healthcare, and governance mechanisms. Wherever access improves, freedom grows.

\section*{Freedom and the Commons}

Because intelligence is enacted by networks, freedom is inseparable from the commons. Choice diversity does not emerge from isolated agents but from shared, maintained substrates. Likewise, the ability to choose depends on collective investments in accessibility, verification, and coordination.

This produces a non-libertarian but non-authoritarian conclusion:
\begin{itemize}
    \item Freedom does not arise \emph{despite} shared structures.
    \item Freedom arises \emph{through} well-designed shared structures.
\end{itemize}

When commons are degraded, monopolized, or rendered opaque, freedom contracts—not because rules exist, but because access collapses and choice-spaces narrow.

\section*{Freedom as a Network Property}

Freedom is therefore not a property of isolated will, but of network topology. It scales with:
\begin{itemize}
    \item redundancy rather than single points of failure,
    \item diversity rather than enforced uniformity,
    \item low barriers rather than high thresholds,
    \item transparent rules rather than arbitrary power.
\end{itemize}

In this sense, freedom is neither absolute nor individual. It is emergent, reciprocal, and conditional. Each agent’s freedom depends on others maintaining the structures that keep choice available, and each agent’s actions reshape the freedom landscape that follows.

\section*{A Reframed Definition of Freedom}

Under forced free will, freedom is not the absence of constraint, but:

\begin{quote}
\emph{The degree to which an agent has accessible, diverse, and actionable choices within a shared constraint landscape that it simultaneously inhabits and helps to shape.}
\end{quote}

Freedom grows when necessities are made easier, clearer, and more accessible—because every reduction in existential friction expands the space of possible futures.

\end{document}
