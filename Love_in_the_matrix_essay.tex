\documentclass[12pt,a4paper]{article}
\usepackage[utf8]{inputenc}
\usepackage[margin=1in]{geometry}
\usepackage{csquotes}
\usepackage{parskip}
\usepackage{titlesec}

% Section formatting
\titleformat{\section}{\large\bfseries}{\thesection}{1em}{}
\titleformat{\subsection}{\normalsize\bfseries}{\thesubsection}{1em}{}

\title{\textbf{Why Love Fades (And What It's Actually Telling You)}\\
\large Your Relationship Is an Economy, and Someone Might Be Strip-Mining It}

\author{}
\date{}

\begin{document}

\maketitle

\begin{center}
\textit{An essay about why ``we just fell out of love'' often means\\``the shared systems we depend on quietly collapsed.''}
\end{center}

\vspace{0.5cm}

\noindent You've probably heard it before, or maybe you've said it yourself:

``We just fell out of love. The spark died. I don't know what happened.''

But what if love isn't something you \textit{maintain}, like keeping a candle lit, but rather \textbf{feedback} that tells you whether the relationship is actually working?

What if ``falling out of love'' isn't a mysterious emotional failure, but your body's accurate signal that someone---maybe you, maybe them, maybe both---has been quietly taking more than they give back?

Here's the reframe:

\begin{quote}
\textbf{A relationship is an economy. Love is the feedback signal that tracks whether both partners are investing in the commons, or whether one (or both) is extracting without replenishing.}
\end{quote}

This isn't a metaphor. It's the same underlying dynamics that govern any system where two entities depend on shared resources---from marriages to economies to ecosystems.

And once you see it, a lot of relationship patterns that seemed mysterious start making uncomfortable sense.

\section{The commons in a relationship}

When two people form a relationship, they create a \textbf{shared system} they both depend on.

This system includes things like:

\textbf{Physical commons:}
\begin{itemize}
    \item A clean, functional living space
    \item Meals planned and prepared
    \item Bills paid, household tasks managed
    \item Shared belongings maintained
\end{itemize}

\textbf{Emotional commons:}
\begin{itemize}
    \item A sense of being valued and appreciated
    \item Emotional labor---listening, caring, remembering, supporting
    \item Conflict resolved rather than accumulated
    \item Trust built and maintained
\end{itemize}

\textbf{Temporal commons:}
\begin{itemize}
    \item Quality time together
    \item Shared experiences and memories
    \item Coordinated schedules and plans
\end{itemize}

\textbf{Developmental commons:}
\begin{itemize}
    \item Both partners having freedom to grow, learn, pursue interests
    \item Sufficient autonomy alongside connection
    \item Space to be yourself, not just half of ``us''
\end{itemize}

These aren't luxuries. They're the \textbf{shared infrastructure} that makes the relationship livable and sustainable.

When this commons is well-maintained, the relationship hums. You both have capacity. You both feel valued. There's slack in the system for stress, conflict, bad weeks.

When the commons degrades, everything gets harder. Small conflicts become big ones. Resentment builds. One or both partners starts feeling depleted, unappreciated, trapped.

And here's the key insight:

\begin{quote}
\textbf{That feeling---that depletion, that resentment, that loss of affection---isn't irrational. It's accurate feedback that the economic balance has broken.}
\end{quote}

\section{Viability needs vs. extraction}

Just like in any economy, there's a crucial distinction between:

\textbf{Legitimate viability needs}---what each partner requires to:
\begin{itemize}
    \item stay mentally and physically healthy,
    \item maintain their sense of self,
    \item have time to rest, reflect, develop,
    \item contribute sustainably over the long term.
\end{itemize}

\textbf{Extraction}---taking more than you need in ways that:
\begin{itemize}
    \item consistently leave your partner compensating,
    \item degrade the shared commons,
    \item accumulate as silent resentment,
    \item create asymmetry that becomes normal.
\end{itemize}

The tricky part? From the inside, extraction often feels like viability needs.

\textbf{Examples of what extraction looks like:}

You leave dishes in the sink ``just this once''---but ``just this once'' happens four times a week, and your partner silently does them while you relax.

You get to pursue your hobbies, see friends, develop your career---while your partner's time is quietly absorbed managing the household, the kids, the social calendar, the mental labor of keeping everything running.

You vent about your day, your stress, your problems---and your partner listens, supports, manages your emotions. But when they need the same, you're tired, distracted, or you turn it into advice rather than presence.

You get to be messy, forgetful, ``not good at that stuff''---and the gap is filled by your partner's extra labor, which becomes invisible because it's constant.

You make plans that assume your partner's flexibility---but when they want the same freedom, there's suddenly a logistical problem.

From your perspective, you're just living your life. From the network perspective, you're quietly mining the commons.

And here's what makes it insidious: \textbf{extraction compounds}.

Every time your partner compensates for your extraction, they:
\begin{itemize}
    \item use capacity they could have spent on themselves,
    \item accumulate a little resentment,
    \item normalize the imbalance,
    \item make it easier for you to keep extracting without noticing.
\end{itemize}

The person doing the extracting often has no idea. They just think, ``This is how our relationship works. They're better at that stuff. They don't mind.''

Meanwhile, the person compensating might not even articulate it to themselves for years. They just slowly feel more tired, more resentful, more \textit{used}.

\section{Attraction vs. love: different feedback mechanisms}

Now let's talk about what biology is actually doing here.

\textbf{Attraction} is an initial assessment mechanism. It asks:
\begin{itemize}
    \item Does this person have qualities I value?
    \item Do they seem capable of investing?
    \item Is there potential here?
\end{itemize}

Attraction gets you interested. It's forward-looking, optimistic, based on signals and potential.

\textbf{Love} is different. Love is a \textbf{sustained response} to how the relationship actually functions over time.

Love asks:
\begin{itemize}
    \item Am I valued and respected?
    \item Is this person investing proportionally in our shared systems?
    \item Do I feel safe, seen, supported?
    \item Is the commons being maintained or degraded?
\end{itemize}

Love \textit{deepens} when the answer to those questions is consistently ``yes.''

Love \textit{fades} when the answer is consistently ``no.''

This isn't about romance or passion or chemistry. Those things matter, but they're not what sustains long-term attachment.

What sustains attachment is the felt sense that \textbf{the system is working---that you're both investing, both benefiting, both keeping the shared infrastructure healthy}.

Love is your body's way of tracking whether this partnership is genuinely mutual or whether you're being gradually depleted.

\section{The ratchet: how relationships compound}

Just like economies, relationships have a \textbf{ratchet effect}.

Every time you and your partner successfully navigate conflict, that pattern becomes part of your shared toolkit. Next time is easier.

Every time one of you shows up for the other in a hard moment, trust accumulates. The foundation gets stronger.

Every inside joke, every shared reference, every ``remember when we...''---these are patterns that make the relationship richer and more resilient over time.

\textbf{This is the positive ratchet.} Good patterns compound. The relationship becomes a stable platform that makes both partners more capable.

But there's also a \textbf{negative ratchet}.

Every time one partner extracts without replenishing, and the other compensates, that imbalance becomes a little more normal.

Every unresolved resentment adds to the pile.

Every time you think ``I'll bring it up later'' and then don't, the gap between the real state of the relationship and what you're willing to acknowledge grows.

Eventually, you're living in a relationship where the commons has quietly collapsed, but neither of you has said it out loud.

And by the time you do, there's a huge backlog of resentment, depletion, and broken trust to work through.

\section{Why ``I just don't love them anymore'' is often accurate feedback}

Here's the thing people don't want to hear:

When you say ``I don't love them anymore,'' you're often describing \textit{accurate biological feedback}.

It might mean:

\textbf{``I've been extracting, and my body knows it.''}

You've been taking more than you give. Your partner has been compensating. And some part of you registers the guilt, the asymmetry, the fact that you're not showing up as an equal partner. That discomfort manifests as emotional distance, loss of affection, wanting out.

\textbf{``They've been extracting, and my body is protecting me.''}

They've been taking more than they give. You've been compensating until you're depleted. And your emotional system, correctly, is withdrawing investment. The love fading isn't a failure---it's your body telling you to stop pouring energy into a system that's draining you.

\textbf{``We're both extracting, and the commons collapsed.''}

Neither of you has been maintaining the shared infrastructure. You've both been running on autopilot, taking what you need in the moment without tracking the cumulative cost. The relationship became a race to extract before the other person does.

In all three cases, ``falling out of love'' isn't mysterious. It's your body's distributed monitoring system telling you the truth:

\begin{quote}
\textbf{This partnership is not functioning as a mutual investment system. Someone is mining it, and the shared foundation is degrading.}
\end{quote}

\section{What healthy investment actually looks like}

So what does a relationship look like when both partners are genuinely investing in the commons?

\textbf{Maintenance is shared and visible}

Housework isn't 70/30. Emotional labor isn't 80/20. Social planning isn't all on one person. Both partners can articulate what the other contributes, because they're paying attention.

\textbf{Freedom flows both ways}

If one partner gets three hours a week for hobbies, gym, friends, personal development---the other gets equivalent space. Not identical activities, but equivalent \textit{freedom}.

\textbf{Emotional labor is reciprocal}

You vent, they listen---and they vent, you listen. You ask about their day, they ask about yours. When one of you is struggling, the other shows up---and vice versa.

\textbf{Both partners actively ask: ``Am I extracting right now?''}

This is the hardest one, because extraction is usually invisible from the inside.

But in healthy relationships, both people cultivate the ability to notice:
\begin{itemize}
    \item Am I leaving work for my partner that I could do myself?
    \item Am I taking freedom without giving equivalent freedom?
    \item Am I using their capacity to manage my stress, my mess, my disorganization---without replenishing theirs?
    \item Is the balance genuinely mutual, or am I rationalizing asymmetry?
\end{itemize}

\textbf{Conflict is maintenance, not failure}

Healthy relationships don't avoid conflict. They treat it as \textbf{commons maintenance}.

When something isn't working, you surface it. You negotiate. You adjust.

Conflict is how you prevent resentment from accumulating silently until the whole system collapses.

\section{The developmental commons: why autonomy matters}

Here's something that often gets missed:

A good relationship doesn't just maintain shared infrastructure. It also protects and \textbf{invests in each partner's individual development}.

You're not just ``us.'' You're also still \textit{you}.

If one partner's sense of self slowly dissolves into the relationship---if their interests, friendships, ambitions, and autonomy quietly erode---that's not romantic devotion. That's the commons failing in a different way.

Both partners need:
\begin{itemize}
    \item time and space to develop their own interests,
    \item relationships and communities outside the partnership,
    \item the freedom to grow in ways that might not directly serve ``us,''
    \item support for becoming more capable, more themselves, over time.
\end{itemize}

This isn't selfish. It's how you keep both partners resourceful and interesting.

A relationship where one person's development is quietly sacrificed for the other's convenience is a relationship where someone is extracting.

And eventually, the person whose self got eroded will either:
\begin{itemize}
    \item realize what happened and feel deep resentment,
    \item or lose themselves so completely they don't even recognize the loss.
\end{itemize}

Neither is sustainable.

\section{What this means for you}

If you're in a relationship---or thinking about one---here's the uncomfortable question this perspective forces:

\begin{quote}
\textbf{Are you investing in the commons, or are you quietly mining it?}
\end{quote}

And the follow-up:

\begin{quote}
\textbf{Is your partner investing, or are they mining you?}
\end{quote}

Because here's what the network view reveals: you can't fake this long-term.

Your body tracks it. Their body tracks it.

Love deepens when the monitoring system detects genuine mutual investment.

Love fades when the monitoring system detects extraction---even if neither of you is consciously admitting it.

The person who's been compensating will eventually deplete. The trust will erode. The resentment will compound.

And by the time you're sitting across from each other saying ``I don't know what happened,'' what happened was actually quite simple:

The economics broke. Someone was extracting more than they gave back. The commons degraded. And love---doing its job as a feedback signal---faded in response.

\section{The hard truth}

Here's the thing nobody wants to say:

If your partner says ``I don't love you anymore,'' there's a decent chance they're telling you something true about the state of the system.

Maybe they've been extracting, and their body is signaling guilt and avoidance.

Or maybe you've been extracting, and their body is correctly withdrawing investment from a system that's depleting them.

Either way, ``I don't love you anymore'' isn't the problem. It's the \textbf{symptom}.

The problem is that somewhere along the way, the mutual investment broke down.

And the feedback signal is telling you both the truth.

\section{Love is distributed monitoring}

We like to think of love as a feeling we choose, a commitment we make, a romantic ideal we strive toward.

But from a biological perspective, love is simpler and more mechanical than that:

\begin{quote}
\textbf{Love is your body's way of monitoring whether this partnership is functioning as a genuinely mutual system.}
\end{quote}

It tracks:
\begin{itemize}
    \item whether you're valued,
    \item whether the work is shared,
    \item whether your capacity is being replenished or drained,
    \item whether the commons is maintained or collapsing.
\end{itemize}

When the system works, love deepens.

When the system fails, love fades.

Not because you're shallow or because relationships are disposable, but because \textbf{your body is correctly telling you that the underlying dynamics are broken}.

This doesn't mean every relationship should last forever.

It means that when love fades, it's worth asking:
\begin{itemize}
    \item What's the actual state of our commons?
    \item Who's been extracting?
    \item Can we rebuild mutual investment?
    \item Or is this system genuinely unsalvageable?
\end{itemize}

Sometimes the answer is: rebuild.

Sometimes the answer is: leave.

But either way, the feedback is telling you something true.

\section*{In one sentence}

If you want the core insight:

\begin{quote}
\textbf{Love isn't something you maintain---it's feedback that emerges when both partners genuinely invest in the shared systems they depend on.}
\end{quote}

When it fades, it's usually telling you the truth:

Someone stopped investing.

The commons degraded.

And your body knows.

\end{document}