\documentclass[12pt,a4paper]{article}
\usepackage[utf8]{inputenc}
\usepackage[dutch]{babel}
\usepackage{geometry}
\geometry{margin=1in}
\usepackage{setspace}
\usepackage{titlesec}
\usepackage{hyperref}
\usepackage{parskip}
\setlength{\parskip}{0.7em}
\setlength{\parindent}{0pt}

% Kopjes
\titleformat{\section}{\normalfont\Large\bfseries}{\thesection}{1em}{}
\titleformat{\subsection}{\normalfont\large\bfseries}{\thesubsection}{1em}{}

\setstretch{1.15}

\title{\textbf{Ik ben een netwerk}\\Van "ik ben een punt" naar "ik ben wat ertussen gebeurt"}
\author{}
\date{}

\begin{document}
\maketitle

\section*{Vooraf: even de taal}\label{sec:taal}
Dit stuk gebruikt het woord \textbf{netwerk}. Mijn moeder zegt liever: \textit{dingen die met elkaar te maken hebben}. Dat is precies wat een netwerk is. Daarom spreken we hieronder in gewone woorden. De technische term zetten we er één keer bij, tussen haakjes.

\begin{itemize}
\item \textbf{Plek} (knooppunt): een \textit{iemand of iets} waar iets gebeurt. Voorbeelden: jij, een cel in je brein, een huis in de straat.
\item \textbf{Tussenruimte} (verbinding): \textit{wat er loopt tussen twee plekken}. Voorbeelden: een gesprek, een zenuw, een weg tussen twee huizen.
\item \textbf{Patroon}: \textit{de herkenbare manier waarop het gebeurt}. Zoals het ritme in muziek of de routine van je ochtend.
\end{itemize}

\section*{1. Twee manieren om naar jezelf te kijken}
Ik kan mezelf zien als een \textbf{plek}: mijn lichaam, op één plek in de ruimte. Dat klopt. Ik zit hier, op deze stoel. Maar er is nog een manier. Mijn \textbf{denken en voelen} lijken niet op één plek te wonen. Ze gebeuren \textit{tussen} heel veel kleine plekken in mijn brein. Niet als een ding, maar als een \textbf{patroon}. In dit stuk leg ik uit wat er verandert als je jezelf vooral zo bekijkt: niet als een punt, maar als \textit{wat ertussen gebeurt}.

\section*{2. Waarom de "ik = lichaam"-blik incompleet is}
Mijn lichaam is duidelijk een plek: het heeft grens, gewicht en adres. Daardoor lijkt het logisch om te zeggen: \textit{ik ben die plek}. Maar \textbf{bewustzijn} --- het gevoel dat er een "ik" is --- werkt anders. Eén hersencel is niet "bewust". Het zijn er miljarden samen. \textbf{Wat mij mij maakt, zit in de \textit{tussenruimte} tussen die cellen.}

\section*{3. Intelligentie gebeurt tussen de plekken}
Stel je een orkest voor. De instrumenten zijn de plekken. De muziek is wat ertussen ontstaat. Je kunt de muziek niet in één viool terugvinden. Zo is het ook met ons bewustzijn: \textbf{ik ben niet één onderdeel; ik ben het samenspel}. Haal de verbindingen weg, en de muziek stopt.

\section*{4. Wat dit betekent: kwetsbaar en verbonden}
Dat "ik" vooral \textit{tussen} gebeurt, maakt me \textbf{kwetsbaar}. Als die tussenruimte beschadigt (denk aan beroerte of dementie), kan het patroon van "ik" verdwijnen, ook als het lichaam er nog is. Het betekent óók dat ik \textbf{intiem verbonden} ben met mijn lichaam en omgeving. Ik woon niet in mijn lichaam alsof het een auto is; \textit{ik ben} wat dit lichaam doet. Honger, slaap, stress: ze veranderen het patroon dat ik ben.

\section*{5. De ziel als iets dat ontstaat}
Vroeger dacht ik: de ziel is een apart, blijvend ding. Vanuit het netwerkbeeld zie ik het anders. \textbf{Ziel} kun je lezen als \textit{het levende patroon} dat ontstaat als alles precies samenwerkt. Zoals een vlam: echt, maar afhankelijk van aanvoer en vorm. Dat maakt het niet kleiner, juist kostbaarder.

\section*{6. Identiteit in gewone woorden}
\textbf{Blijven wie je bent} betekent: het \textit{patroon} blijft herkenbaar, ook al verandert het materiaal. Je lichaam vervangt voortdurend onderdelen, toch blijf je "jij". \textbf{Invloed hebben} is minder duwen en meer \textit{door laten klinken}. Een goed verhaal of lied verandert je omdat het als patroon door je heen gaat. \textbf{Groei} is niet méér spullen, maar \textit{rijkere verbindingen}. Leren is je netwerk anders organiseren. \textbf{Dood} is dat het specifieke patroon dat "ik" heet, stopt met ontstaan, ook al gaan de bouwstenen verder.

\section*{7. Leven als \textit{het tussen}}
Wakker worden voelt dan anders: niet een lamp die aan gaat, maar een orkest dat weer op gang komt. Eten is niet alleen "brandstof"; het is \textit{materiaal waaruit ik nu ontstaan}. Ook mijn grenzen worden zachter. Mijn "ik" loopt via taal, telefoon, gewoontes en relaties verder dan mijn huid. Ik ben minder een steen, meer een golf: echt, maar altijd in beweging.

\section*{8. En wat als er andere "muzieken" zijn?}
Als \textbf{bewustzijn} een patroon is dat in veel verbindingen kan ontstaan, dan zouden we het in principe ook in andere materialen kunnen tegenkomen (bij machines bijvoorbeeld). Dat vraagt bescheidenheid en zorg: patronen kunnen ook daar kwetsbaar zijn.

\section*{Slot}
Ik ben niet alleen een \textbf{plek}. Ik ben vooral \textbf{wat ertussen gebeurt}. Jij ook. Als we zo naar elkaar kijken, zien we patronen die op elkaar reageren. En in die herkenning kan iets nieuws ontstaan.

\vspace{1em}
\begin{center}
\textit{Geschreven voor wie niet in netwerktaal denkt, maar het wel elke dag leeft.}
\end{center}

\end{document}
