\documentclass[11pt,a4paper]{article}

% ---------- Packages ----------
\usepackage[utf8]{inputenc}
\usepackage[T1]{fontenc}
\usepackage{lmodern}
\usepackage[a4paper,margin=2.5cm]{geometry}
\usepackage{setspace}
\usepackage{microtype}
\usepackage{xcolor}
\usepackage{graphicx}
\usepackage{amsmath,amssymb}
\usepackage{enumitem}
\setlist{itemsep=4pt,topsep=4pt}
\usepackage{hyperref}
\hypersetup{
  colorlinks=true,
  linkcolor=blue!60!black,
  urlcolor=blue!60!black,
  citecolor=blue!60!black,
  pdftitle={The Universality of the Search},
  pdfauthor={}
}
\usepackage{tikz}
\usetikzlibrary{arrows.meta,positioning,fit,shapes.misc,shapes.geometric}

% ---------- Title ----------
\title{\textbf{The Universality of the Search:\\
Persistence across Evolution, Society, and AI}}
\author{}
\date{\today}

\begin{document}
\maketitle
\onehalfspacing

\begin{abstract}
Existence is not defined by already having stability; it is defined by the ongoing search for stability. 
\emph{The search is the finding.} 
This document presents a cross-domain account of persistence---from biological evolution and human societies to AI alignment---and a unifying diagram. 
Across substrates, systems persist by looping: searching $\rightarrow$ attaining local stability $\rightarrow$ facing perturbations $\rightarrow$ resuming the search. 
Stability is thus an emergent, local outcome of the search itself, not its termination.
\end{abstract}

\section*{The Universality of the Search}

\subsection*{Premise}
Existence is not defined by already having stability. 
It is defined by the ongoing search for stability. 
\emph{The search is the finding.} 
Persistence itself is nothing more and nothing less than this loop of continuous searching. 
This principle applies across domains: in biological evolution, in human societies, and in the alignment of artificial intelligence.

\subsection*{Evolution as Persistent Search}
Evolution is not a finished product but a continual process. 
Species persist because they are always searching for stability: in food sources, habitats, cooperative strategies, and reproductive success.  
The mechanisms of mutation and natural selection are not end-goals; they are search strategies. 
What emerges is not perfection, but \emph{local stability}---a niche in which a lineage can survive long enough to reproduce.  

\begin{quote}
Evolution persists because it never stops searching. 
The search for stability is the stability of life itself.
\end{quote}

\subsection*{Society as Conflicting Search}
Human societies are likewise sustained by ongoing, and often conflicting, searches for stability.  
\begin{itemize}
  \item Migrants search for stability through safety, livelihood, and a future for their children.  
  \item Citizens opposing migration search for stability through identity, security, and predictability.  
\end{itemize}

Both groups are driven by the same principle: persistence through the search for stability.  
Their searches clash, but the clash itself is also a sign of persistence: 
each group maintains itself by insisting on its own form of stability.  

From a third-person perspective, society persists by enabling \emph{both} searches.  
The task is not to erase one, but to manage the \emph{edges} between them: maintain, repair, and translate, so that conflicting searches can coexist without collapse.  

\subsection*{AI Alignment as Search}
The alignment of artificial intelligence with human values cannot be understood as a fixed endpoint.  
It is, like evolution and society, an ongoing search.  
\begin{itemize}
  \item Human values evolve and shift.  
  \item AI systems adapt, learn, and explore vast solution spaces.  
  \item Alignment is the persistence of the search for compatibility, not a one-time solution.  
\end{itemize}

Just as evolution survives by constant adaptation, 
and societies endure through negotiation and repair of conflicts, 
so too must alignment be treated as a loop.  
\emph{Alignment is persistence of the search itself.}

\subsection*{The General Principle}
Across all substrates, the same dynamic appears:
\begin{itemize}
  \item In physics: systems relax along paths of least resistance toward \emph{local} minima.  
  \item In biology: species adapt by searching for niches that provide viability.  
  \item In society: groups persist by seeking stability, even in conflict.  
  \item In AI: safety emerges not from closure but from continuous alignment-seeking.  
\end{itemize}

\begin{quote}
\textbf{Axiom (Universality of the Search).}  
Persistence across domains is identical with the ongoing search for stability.  
The search itself is already the stability that sustains existence.
\end{quote}

\subsection*{Implications}
This principle reframes how we view systems:  
\begin{enumerate}[label=\textbf{\arabic*.}]
  \item \textbf{No absolute stability.} Stability is always local and temporary. The loop continues.  
  \item \textbf{Conflict as persistence.} Even opposing searches are part of persistence; the clash is itself a mode of survival.  
  \item \textbf{Alignment as loop.} Safety in technology, like viability in biology or society, is not solved but maintained.  
  \item \textbf{Universality.} From atoms to societies to AI, the same structure applies: the search sustains the loop, and the loop is existence.  
\end{enumerate}

\subsection*{Conclusion}
What appears as difference---evolution, society, artificial intelligence---is in fact one and the same process:  
\emph{the search for stability that itself constitutes stability.}  
This is the loop of persistence. 
It is not a destination, but an act. 
It is what keeps life alive, societies together, and intelligence---human or artificial---aligned with existence.  

\section*{Figure: Three Loops, One Principle}

\begin{figure}[ht]
  \centering
  \tikzset{
    box/.style={draw, rounded corners, align=center, inner sep=5pt, minimum width=2.6cm, font=\small, fill=white},
    titlebox/.style={draw, rounded corners, align=center, inner sep=6pt, font=\bfseries, fill=blue!5},
    lbl/.style={font=\footnotesize, align=center},
    arrow/.style={-{Latex}, thick},
    light/.style={draw, rounded corners, fill=gray!8, inner sep=6pt},
    bubble/.style={ellipse, draw, align=center, inner sep=4pt, font=\footnotesize, fill=white}
  }

  % Central principle
  \node[titlebox] (center) {Search Loop (Persistence)\\\footnotesize\emph{The search is the finding}};

  % Evolution loop (top-left)
  \node[light, above left=2.0cm and 3.5cm of center] (Epanel) {\parbox{7.6cm}{
      \centering \textbf{Evolution}\\[2mm]
      \begin{tikzpicture}[baseline=(current bounding box.center), node distance=9mm]
        \node[box] (E1) {Search / Variation};
        \node[box, right=of E1] (E2) {Local Stability\\(Niche)};
        \node[box, right=of E2] (E3) {Perturbation};
        \node[box, right=of E3] (E4) {Resume Search};
        \draw[arrow] (E1)--(E2);
        \draw[arrow] (E2)--(E3);
        \draw[arrow] (E3)--(E4);
        \draw[arrow] (E4) .. controls +(0,-1.2) and +(0,-1.2) .. (E1);
      \end{tikzpicture}
  }};

  % Society loop (bottom-left)
  \node[light, below left=2.0cm and 3.5cm of center] (Spanel) {\parbox{7.6cm}{
      \centering \textbf{Society (Conflicting Searches)}\\[2mm]
      \begin{tikzpicture}[baseline=(current bounding box.center), node distance=8mm]
        \node[box] (S1) {Search A:\\Migrants};
        \node[box, right=2.2cm of S1] (S2) {Edge\\Maintenance};
        \node[box, right=2.2cm of S2] (S3) {Search B:\\Anti-migrants};
        \draw[arrow] (S1)--(S2);
        \draw[arrow] (S3)--(S2);
        \node[bubble, above=7mm of S2] (S4) {Third-person\\translation/repair};
        \draw[arrow] (S2)--(S4);
        \draw[arrow] (S4) .. controls +(0,1.0) and +(0,1.0) .. (S1);
        \draw[arrow] (S4) .. controls +(0,1.0) and +(0,1.0) .. (S3);
      \end{tikzpicture}
  }};

  % AI loop (right side)
  \node[light, right=3.8cm of center] (Apanel) {\parbox{7.6cm}{
      \centering \textbf{AI Alignment}\\[2mm]
      \begin{tikzpicture}[baseline=(current bounding box.center), node distance=9mm]
        \node[box] (A1) {Search:\\Human \& AI};
        \node[box, right=of A1] (A2) {Local Compatibility};
        \node[box, right=of A2] (A3) {Drift/Shift\\(Values/Models)};
        \node[box, right=of A3] (A4) {Resume Search};
        \draw[arrow] (A1)--(A2);
        \draw[arrow] (A2)--(A3);
        \draw[arrow] (A3)--(A4);
        \draw[arrow] (A4) .. controls +(0,-1.2) and +(0,-1.2) .. (A1);
      \end{tikzpicture}
  }};

  % Connect panels to center
  \draw[arrow] (center) -- node[lbl,above left] {applies to} (Epanel);
  \draw[arrow] (center) -- node[lbl,below left] {applies to} (Spanel);
  \draw[arrow] (center) -- node[lbl,above] {applies to} (Apanel);

  % Legend
  \node[below=1.2cm of center, align=left, font=\footnotesize] (legend) {%
    \textbf{Legend:} \\
    \(\rightarrow\) arrows denote loop progression (search $\to$ local stability $\to$ perturbation $\to$ search).\\
    Edge maintenance / translation keeps conflicting searches from collapsing.};

\caption{Three domains, one principle. Each domain persists by looping: searching $\rightarrow$ local stability $\rightarrow$ perturbation $\rightarrow$ renewed searching. 
Stability is an emergent, local outcome of the search itself.}
\label{fig:three-loops}
\end{figure}

\end{document}
