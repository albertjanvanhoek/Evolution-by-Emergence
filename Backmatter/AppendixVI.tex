\chapter{Appendix VI \\ Paradigm-Shift Matrix}

\noindent
\textbf{Purpose.}  The matrix on the next page condenses the conceptual transition that underlies this book.  It juxtaposes a \emph{Classical Atomistic–Mechanistic} worldview—still implicit in much everyday reasoning—with a \emph{Relational–Process Emergent} worldview that informs contemporary network science, developmental biology, systems thinking, and process theology.  

\noindent
\textbf{Scope.}  Nine dimensions are listed: ontology, causation, change, epistemic method, evolution, governance, conflict, ethics, and the religious or theological imaginary.  Each row is deliberately no more than one sentence per column, making the table a quick “switchboard” for readers who want to locate where their current intuitions sit and what would shift if they adopt the new paradigm.

\noindent
\textbf{Use.}  Readers may treat the matrix as (i)~a checklist when evaluating theories; (ii)~a glossary that unpacks terms used throughout the chapters; and (iii)~a springboard for interdisciplinary dialogue, since every row maps onto an active debate in philosophy, social science, and theology.

\bigskip

\begin{table}[ht]
\centering
\caption{Paradigm shift: from a Classical Atomistic--Mechanistic view to a Relational--Process Emergent view}
\label{tab:paradigm-shift}
\begin{tabularx}{\textwidth}{@{}p{3.2cm}X X@{}}
\toprule
\textbf{Dimension} &
\textbf{Classical / Atomistic--Mechanistic} &
\textbf{Relational--Process / Emergent} \\
\midrule
Basic ontology &
Substances first—discrete, self-contained entities with intrinsic properties &
Relations first—entities as nodes in dynamic webs; properties partly constituted by connections \\[0.4em]

Model of causation &
Linear, efficient (\(A \rightarrow B \rightarrow C\)) &
Recursive feedback loops; upward + downward causation; patterns arise from cycles \\[0.4em]

Pattern of change &
External push or design imposed by an outside agent &
Self-organising emergence from local interactions \\[0.4em]

Epistemic method &
Analysis by decomposition; isolate parts, hold context constant &
Interaction mapping; model flows, constraints and real-time adaptation \\[0.4em]

Evolutionary lens &
“Survival of the fittest” (competition among isolated units) &
“Fit of the network” (co-adaptation; links that stabilise the whole persist) \\[0.4em]

Governance style &
Command–control hierarchy; top-down orders &
Distributed adaptive governance; authority node-specific and fluid \\[0.4em]

Conflict grammar &
Blame and retribution; locate fault, punish &
Diagnostic empathy; surface needs, negotiate repair \\[0.4em]

Ethical centre &
Rule-keeping, negative duty (avoid harm) &
Care and resilience; actively maintain relational health \\[0.4em]

Religious / theological imaginary &
Transcendent monarch outside the system; worship = obedience &
Immanent communion within relations; the sacred unfolds through participation and \textit{agapē} \\
\bottomrule
\end{tabularx}
\end{table}
