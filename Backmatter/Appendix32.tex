
\documentclass[11pt,a4paper]{article}

% --------------------------------------
% Packages
% --------------------------------------
\usepackage[utf8]{inputenc}
\usepackage[T1]{fontenc}
\usepackage{lmodern}
\usepackage{geometry}
\geometry{margin=1in}
\usepackage{microtype}
\usepackage{amsmath, amssymb}
\usepackage{graphicx}
\usepackage{hyperref}
\usepackage{xcolor}
\hypersetup{
  colorlinks=true,
  linkcolor=blue,
  urlcolor=blue,
  citecolor=blue
}

% --------------------------------------
% Metadata
% --------------------------------------
\title{\textbf{Autonomous Interdependence: The Two Axes of Viable Freedom}}
\author{Albert Jan van Hoek \\ \small Evolution by Emergence Project}
\date{\today}

% --------------------------------------
% Document
% --------------------------------------
\begin{document}
\maketitle

\begin{abstract}
Freedom is often conflated with independence---the ability to act without constraint.
Yet complete independence is ecologically, socially, and biologically impossible.
Every living system depends on exchanges with others to stay viable.
This paper introduces a dual-axis model of freedom---\emph{autonomy} and \emph{interdependence}---as complementary properties of viable systems.
Together they yield \textbf{autonomous interdependence}: individual agents who self-regulate while remaining reliably coupled to others.
This principle clarifies the paradox of freedom, offers a lens for relational ethics, and provides a generalizable framework for stability across scales (persons, institutions, ecologies, and AI networks).
\end{abstract}

\noindent\textbf{Keywords:} autonomy, interdependence, freedom, viability, emergence, networks, trust, repair

\section{The Paradox of Freedom}
In contemporary discourse, \emph{freedom} is often treated as \emph{doing what I want}.
Pursued in the absolute, one person's freedom quickly becomes another's constraint.
At scale this produces polarization, brittle institutions, and ecological overshoot.
\emph{The paradox:} the pursuit of absolute freedom destroys the conditions that make freedom possible.

A viable account of freedom must include relationship, not only agency.

\section{Freedom as a Dual-Dimensional Construct}
We propose freedom has two orthogonal dimensions:
\begin{itemize}
    \item \textbf{Autonomy} (local): the capacity for self-regulation---to maintain internal stability and act in line with one's values under real constraints.
    \item \textbf{Interdependence} (relational): the capacity for reliable coupling---to exchange information, care, and resources with other autonomous agents.
\end{itemize}

\noindent Failure modes when isolated:
\begin{itemize}
    \item Without autonomy: dependence (loss of self-regulation). With inflated autonomy: isolation (loss of feedback).
    \item Without interdependence: fragmentation (no reciprocity). With unbalanced interdependence: co-dependence or domination.
\end{itemize}

\paragraph{Definition (informal).}
\emph{Autonomous interdependence} is the state in which autonomous agents remain reliably coupled so that each agent's choices increase---not reduce---the viable choices of others.

\section{Formalization (EbE framing)}
Let agent $i$ possess internal viability $V_i$ that depends on both internal regulation $A_i$ and relational coupling $C_{ij}$ with others $j$:
\begin{equation}
  V_i = f\!\left(A_i, \sum_{j} C_{ij}\right).
\end{equation}
Here, $A_i$ captures internal coherence (self-regulation, adaptive choice) and $C_{ij}$ captures reliability, reciprocity, transparency, and repair capacity between $i$ and $j$.

A system expresses \emph{autonomous interdependence} when
\begin{equation}
  \frac{\partial V_i}{\partial A_i} \approx \frac{\partial V_i}{\partial C_{ij}} \quad \text{for the relevant set of } j,
\end{equation}
i.e., when internal and external regulation contribute proportionally to viability.
Too little coupling $\Rightarrow$ isolation collapse; too much coupling $\Rightarrow$ loss of agency.
Sustainable freedom emerges near the balance point.

\section{Psychological and Societal Implications}
\subsection{Individual level}
Autonomy is not isolation; it is self-governance \emph{while connected}: emotional regulation, reflective choice, and responsibility for one's own state.
Interdependence then becomes safe exchange between such individuals.
Relationships thrive when partners are both self-stabilizing and mutually responsive.

\subsection{Collective level}
Institutions mirror the same pattern.
Systems fail when either autonomy (local control) or interdependence (shared regulation) dominates unilaterally: centralization squashes local learning; fragmentation erodes coherence.
Sustainable democracies, ecologies, and AI networks require strong local autonomy and reliable global interdependence.

\section{Freedom Reinterpreted}
We can now define freedom operationally:
\begin{quote}
\textbf{Freedom is the lived capacity to self-regulate while relying on others who reliably self-regulate with you.}
\end{quote}
Freedom is not disconnection; it is \emph{mutual reliability}.
It expands the choice space for all participants, not just one.
In shorthand:
\begin{equation}
  \text{Freedom} \;\propto\; \text{Autonomy} \times \text{Interdependence}.
\end{equation}

\section{Ethical Consequences}
This framework yields practical design commitments:
\begin{enumerate}
    \item \textbf{Education}: train autonomy (self-reflection, self-regulation) before demanding cooperation.
    \item \textbf{Governance}: build interdependence through transparency, explicit commitments, and repair protocols.
    \item \textbf{Technology}: design AI and digital systems that preserve local autonomy while enabling cooperative feedback.
    \item \textbf{Relationships}: evaluate health by \emph{reciprocal viability} rather than dependence/independence.
\end{enumerate}

\section{Fear as the Gate to Freedom}
Fear collapses feedback and narrows choice.
Freedom arises when fear can be named and integrated---when agents face uncertainty without severing connection.
Naming fear converts it from a driver into data; this enables autonomy to act intelligently within interdependence.

\section{Summary Table}
\begin{center}
\begin{tabular}{|l|p{6cm}|p{6cm}|}
\hline
\textbf{Concept} & \textbf{Key property} & \textbf{Failure when isolated} \\ \hline
Autonomy & Self-regulation and agency & Isolation or dependence \\ \hline
Interdependence & Reliable reciprocity (transparent, repairable) & Control, co-dependence, or collapse \\ \hline
Freedom & Viability across both axes & Fragility via loss of feedback \\ \hline
\end{tabular}
\end{center}

\section{Closing Thought}
True freedom is not the absence of constraints, but the presence of reliable others within them---the equilibrium between self and system: autonomy alive inside interdependence.

\vspace{1em}
\noindent\textit{Suggested citation:} van Hoek, A. J. (2025). \emph{Autonomous Interdependence: The Two Axes of Viable Freedom.} Evolution by Emergence Project (version \today).

\end{document}
