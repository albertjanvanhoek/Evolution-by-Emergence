\documentclass[11pt,a4paper]{article}

% ---------- Packages ----------
\usepackage[utf8]{inputenc}
\usepackage[T1]{fontenc}
\usepackage{lmodern}
\usepackage[a4paper,margin=2.5cm]{geometry}
\usepackage{setspace}
\usepackage{microtype}
\usepackage{xcolor}
\usepackage{hyperref}
\hypersetup{
  colorlinks=true,
  linkcolor=blue!60!black,
  urlcolor=blue!60!black,
  pdftitle={Manifesto for an Edge-Centric Society},
  pdfauthor={}
}

% ---------- Title ----------
\title{\textbf{Manifesto for an Edge-Centric Society:\\
Conscious Loops, Cognitive Hygiene, and Evolution by Emergence}}
\author{}
\date{}

\begin{document}
\maketitle
\onehalfspacing

\section*{Preamble}
We live in loops. Each person, each community, each democracy continuously cycles through input, labelling, checking, updating, decision, action, and feedback. Consciousness itself is this loop: the act of checking and updating our world models. But under stress, checks weaken, errors slip through, and our models fill with \emph{mud}. Societies, too, can be muddied—when fear, haste, or manipulation bypass the checks that keep collective knowledge clean.

\section*{Our Insight}
If individuals and communities become \emph{aware of the loop}, they can guard against mud. By slowing down under stress, insisting on evidence, and repairing edges of trust, they preserve the integrity of their models. When societies embed this awareness into their institutions, democracies become resilient. And when communities align with the principle of \emph{Evolution by Emergence} (EbE)—``keep alive what keeps us alive''—we gain a compass for sustainable collaboration.

\section*{Our Principles}
\begin{enumerate}
  \item \textbf{Edges First:} The quality of relationships matters more than central nodes. Healthy links create resilience.
  \item \textbf{Cognitive Hygiene:} Recognize stress, slow updates, strengthen checks, and filter mud.
  \item \textbf{Polycentric Democracy:} Many overlapping centers of decision-making, no single point of failure.
  \item \textbf{Transparency and Repair:} Open provenance of information, rituals of forgiveness, and rapid edge repair.
  \item \textbf{Viability First:} Decisions must serve life, resilience, and long-term flourishing.
\end{enumerate}

\section*{Our Commitment}
We commit to building \emph{edge-centric, decentralized communities}: autonomous cells, richly connected, practicing loop-awareness and cognitive hygiene, guided by EbE’s viability-first ethos. Local autonomy with global coherence. Diversity with redundancy. Emergence with resilience.

\section*{Closing}
This manifesto is not a utopia. It is a protocol: a way to live, decide, and build together, so that both individuals and societies remain viable. By caring for our loops and our edges, we keep alive what keeps us alive.

\end{document}
