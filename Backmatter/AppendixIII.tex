% ------------------------------------------------------------------
%  Appendix I+ – SCAP Implementation Notes
%  (To be \input{} into the main Overleaf project: no standalone preamble.)
% ------------------------------------------------------------------

\chapter{Appendix I$^{+}$\\Implementation Notes for the Sustainable Collaborative Alignment Protocol (SCAP)}\label{appendix:scap}

% Register in global TOC (assuming \tableofcontents lives upstream)
\addcontentsline{toc}{chapter}{Appendix I$^{+}$ — SCAP Implementation Notes}

% =======================================================
\section*{Preface}
\addcontentsline{toc}{section}{Preface}
\paragraph{Purpose.}\ This appendix complements Appendix~I (\emph{A Sustainable Collaborative Alignment Protocol}) by providing concise implementation notes—templates, algorithms, and metrics—for practitioners who wish to experiment with SCAP in real projects.

\paragraph{Structure.}\ Each section maps to one or more of the nine SCAP building blocks (A–I) and follows a fixed rhythm:
\begin{itemize}
  \item \textbf{Essence}: one‑sentence distilled idea.
  \item \textbf{Rationale}: why it matters for network resilience.
  \item \textbf{Implementation Notes}: checklists, policy handles, or algorithms.
  \item \textbf{References \& Templates}: where to dive deeper.
\end{itemize}
Fork, remix, and contribute back via the project repository.

\bigskip\hrule\bigskip

% =======================================================
\section{Owner–Steward Duality (Blocks A \& G)}
\subsection*{Essence}
\emph{"It’s mine \emph{and} ours." Ownership grants control; stewardship guards continuity.}

\subsection*{Rationale}
Markets reward decisive owners, yet complex systems persist only when the commons is actively tended.  A dual legal wrapper—`owner \emph{plus} steward'—reconciles both incentives.

\subsection*{Implementation Notes}
\begin{itemize}
  \item Declare a measurable \textbf{stewardship objective} (soil carbon \%, uptime, etc.).
  \item Link voting or profit rights to \emph{ongoing} compliance with that objective.
  \item Nominate an \textbf{emergency trustee} empowered to claw back assets on gross breach.
  \item Publish an annual \textbf{impact audit} signed by an independent reviewer.
\end{itemize}

\subsection*{Minimum‑Viable Clause}
\begin{quote}\itshape
Rights to use, transfer, or profit from this asset are conditional on maintaining (or improving) its health, measured against the shared metrics in Section~\ref{sec:metrics}.  Failing that, control reverts to the commons trustee.
\end{quote}

\subsection*{References \& Templates}
\begin{itemize}
  \item Purpose Foundation \texttt{Steward‑Ownership} clauses.
  \item UK \texttt{Community‑Interest Company} model (dividend cap 35\%).
  \item ICA \texttt{Worker/User Co‑operative} bylaws.
\end{itemize}

\bigskip\hrule\bigskip

% =======================================================
\section{Commons Protocol Engineering (Blocks B, C \& I)}\label{sec:protocol}
\subsection*{Essence}
\emph{Rough consensus \& running code.} Governance evolves like software: version, test, iterate.

\subsection*{Rationale}
Open standards lower coordination costs across a network. Clear versioning and conformance testing prevent fragmentation while permitting innovation.

\subsection*{Implementation Notes}
\begin{enumerate}
  \item \textbf{Draft v0.x}: open work‑group; public issue tracker.
  \item \textbf{Pilots}: at least two interoperable reference implementations.
  \item \textbf{Release v1.0}: governance vote with super‑majority.
  \item \textbf{SemVer upgrades}: backward‑compatible minors (\texttt{1.x}); breaking changes major (\texttt{2.0}).
\end{enumerate}

\subsection*{Examples}
\begin{itemize}
  \item \textbf{Road‑lane grammar}: ISO 3864 symbols + regional profiles.
  \item \textbf{Health‑data grammar}: HL7 FHIR R5 (2023).
  \item \textbf{OpenAg Warehousing}: ERC‑20 + GS1 DID proof‑of‑origin.
\end{itemize}

\subsection*{References}
RFC 7282 (rough consensus); \href{https://hl7.org/fhir}{FHIR R5}; ISO 3864.

\bigskip\hrule\bigskip

% =======================================================
\section{Economic Surplus Without Rent (Blocks D \& E)}
\subsection*{Essence}
\emph{Surplus funds resilience; rent feeds entropy.}

\subsection*{Rationale}
Margin keeps the lights on; surplus renews the asset; rent—windfall unearned by value creation—must cycle back to the commons or it ossifies the network.

\subsection*{Implementation Notes}
\begin{center}
\begin{tabular}{@{}lll@{}}
\toprule
\textbf{Vehicle} & \textbf{Cap Rule} & \textbf{Typical Surplus Use} \\
\midrule
Steward‑ownership & 0\,\% external dividend & Re‑invest / donate \\
CIC (UK) & 35\,\% dividend cap & Community grants \\
Co‑operative & Patronage refund & Member equity, rebates \\
\bottomrule
\end{tabular}
\end{center}

\subsection*{References}
Elinor Ostrom, \emph{Governing the Commons}; Purpose Foundation toolbox.

\bigskip\hrule\bigskip

% =======================================================
\section{Digital Senescence \& Hardware Rotation (Block F)}
\subsection*{Essence}
\emph{Old nodes retire so the network stays young.}

\subsection*{Rationale}
Scheduled churn prevents stagnation, encourages newcomers, and aligns hardware lifecycles with ecological limits.

\subsection*{Implementation Notes — Churn Algorithm}
Stake collateral $S$ decays at rate $\lambda$. If performance $<P_{\min}$ or stake $<S_{\min}$:
\begin{enumerate}
  \item Node publishes encrypted state snapshot.
  \item Stake reclaimed minus exit audit fee.
  \item Scheduler on‑boards the highest‑rank newcomer.
\end{enumerate}

\subsection*{Analogy}
Apoptosis prunes damaged cells; digital senescence prunes stale hardware.

\bigskip\hrule\bigskip

% =======================================================
\section{Luxury as Conspicuous Stewardship (Block H)}
\subsection*{Essence}
\emph{Signal status by funding the commons, not by externalising costs.}

\subsection*{Rationale}
When prestige spending internalises ecological impact, high‑consumption lifestyles can become net contributors to public goods.

\subsection*{Implementation Notes — Stewardship Points (SP)}
\begin{itemize}
  \item $1$ SP = 1 verified hour of commons service \textit{or} removal of 10 kg CO$_2$.
  \item SPs decay 3\,\% per year to discourage hoarding.
  \item Registering a car emitting $>$200 g km$^{-1}$ costs 10 000 SP.
  \item Essential goods (basic mobility, data, healthcare) carry zero SP‑burn.
\end{itemize}

\bigskip\hrule\bigskip

% =======================================================
\section{Longevity, Fertility \& Lifetime Eco‑Budgets}
\subsection*{Essence}
\emph{More years need fewer tonnes.}

\subsection*{Rationale}
Demographic shifts and life‑extension therapies alter per‑capita resource trajectories. Lifetime eco‑budgets cap total externalities rather than annual flow.

\subsection*{Numbers to Watch}
\begin{itemize}
  \item Global fertility 2024: 2.2; peak population ≈ 2060–65 (UN WPP 2024).
  \item Escape‑velocity therapies in trial (Altos Labs 2024; Calico 2023).
\end{itemize}

\subsection*{Implementation Notes — Policy Handles}
\begin{enumerate}[label=\alph*)]
  \item Lifetime \textbf{ecological budget} per citizen (CO$_2$e, land, phosphorous).
  \item \textbf{Role rotation}: authority seats auto‑expire every $N$ years.
  \item \textbf{Longevity royalty}: levy on patents/services extending lifespan $>$120 yr, earmarked for ecosystem restoration.
\end{enumerate}

\bigskip\hrule\bigskip


% =======================================================
\section{Metrics \& Dashboards}
\label{sec:metrics}
\subsection*{Essence}
\emph{What gets openly measured can be collectively improved.}

\subsection*{Key Indicators}
\begin{itemize}
  \item \textbf{Soil health}: percentage organic matter, infiltration rate (mm·h$^{-1}$).
  \item \textbf{Surface-water quality}: nitrate ($\le$10 mg·L$^{-1}$) and phosphate ($\le$0.1 mg·L$^{-1}$).
  \item \textbf{Biodiversity index}: Shannon diversity $H'$ on sentinel plots.
  \item \textbf{Compute uptime}: rolling SLA $\ge$ 99.9 \% (monthly).
  \item \textbf{Energy mix}: share of renewables $\ge$ 70 \% (annual).
  \item \textbf{Stewardship‑Point ledger}: on‑chain explorer with zk‑proof privacy layer.
\end{itemize}

\subsection*{Open‑Source Tooling}
\begin{description}[style=nextline]
  \item[Grafana + Prometheus] for real‑time metrics and alerting.
  \item[OpenEO] for satellite‑derived soil & biomass monitoring.
  \item[Hyperledger Besu] for SP‑ledger smart contracts (PoS).
  \item[Superset] for public dashboards embedding SQL views.
\end{description}

\subsection*{Implementation Notes}
\begin{enumerate}
  \item Define one canonical JSON‑schema per indicator; validate at ingestion.
  \item Store raw observations; derive indicators via version‑controlled notebooks.
  \item Publish dashboards under CC‑BY with query links for reproducibility.
  \item Run quarterly audits; sign digests on the SP‑ledger.
\end{enumerate}

\bigskip\hrule\bigskip

% =======================================================
\section*{Glossary (Abridged)}
\begin{description}[style=nextline]
  \item[SCAP] Sustainable Collaborative Alignment Protocol.
  \item[Stake] Collateral posted to signal commitment; can be slashed on breach.
  \item[Surplus] Earnings above required margin, earmarked for resilience.
  \item[Rent] Windfall captured through scarcity/monopoly; recycled to commons.
  \item[Stewardship Point (SP)] Scarcity‑backed credit pricing luxury by ecological shadow.
\end{description}

\bigskip\hrule\bigskip

% =======================================================
\section*{References (Selected)}
\addcontentsline{toc}{section}{References (Selected)}
\begin{enumerate}[label=\arabic*.]
  \item UN DESA. \emph{World Population Prospects 2024}.
  \item HL7 International. \emph{FHIR R5 Specification}, 2023.
  \item ISO 3864. \emph{Safety Colours and Safety Signs}, 2019.
  \item Purpose Foundation. “Steward‑Ownership Legal Toolbox,” 2023.
  \item Grafana Labs. \emph{Grafana OSS 10.0 Documentation}, 2024.
\end{enumerate}
