\documentclass[11pt,a4paper]{article}

% --- Packages ---
\usepackage[utf8]{inputenc}
\usepackage[T1]{fontenc}
\usepackage{amsmath}
\usepackage{amsfonts}
\usepackage{amssymb}
\usepackage{geometry}
\geometry{margin=1in}
\usepackage{abstract}
\usepackage{authblk}
\usepackage{hyperref}

% --- Metadata ---
\title{\textbf{The Inevitable Ascent: Selection for Complexity in Persistent Systems}}
\author{Albert Jan van Hoek} % You can change this to your preferred name
\affil{All views and ideas in this paper are my own, developed outside the scope of my employment, and should not be attributed to my employer.}
\date{January 2026}

\begin{document}

\maketitle

\begin{abstract}
How do complex, far-from-equilibrium systems—from the first biological cells to modern artificial intelligence—manage to persist against the constant pull of entropy? In this piece, I propose that there is a constant ``climb'' toward higher complexity. This climb is not a byproduct of chance, but a mathematical necessity for survival. I argue that this persistence requires a specific architecture: a distributed network of monitors that control and influence the system’s progress. This architecture emerges because any structure that emerges adds both efficiency and feedback to the system and is selected based on these two criteria. By viewing complexity through the dual lenses of feedback and efficiency, a self-reinforcing ``closed loop'' appears. I suggest that complex systems must inevitably build layers of internal monitoring to survive, and that this architecture provides a definitive blueprint for AI safety.
\end{abstract}

\section{The Necessity of the Climb}
We often view complexity as a luxury of evolution, but I contend it is a requirement of duration. For any system to exist over time, it must maintain its vital resources—the fundamental resources or conditions required for its existence—above a critical failure threshold. In a chaotic environment, a static defense is a death sentence. To survive, a system must move, grow, and adapt. This creates an inherent pressure to ``climb'' toward higher states of organization. Persistence is not the mere absence of change; it is the active management of vitality.

\section{The Selection Criteria: Efficiency and Feedback}
The mechanism of this climb is rooted in how new traits are selected. When a random mutation or an emergent interaction occurs within a system, its ``value'' is adjudicated by its contribution to two specific functions:

\begin{itemize}
    \item \textbf{Efficiency (Advancement):} Does this new complexity allow the system to process energy better, move faster, or compute more effectively? This represents the system's momentum and its ability to secure more resources.
    \item \textbf{Feedback (Monitoring):} Does this new complexity provide the system with better information about its own safety margins? This represents the system's ability to avoid the ``floor'' of collapse.
\end{itemize}

I have observed that complexity is the currency paid to balance these two needs. A system may emerge with a trait that offers high efficiency but poor feedback; such a system grows quickly but remains fragile, eventually collapsing under the first unforeseen disturbance. Conversely, a system with high feedback but zero efficiency is stagnant and eventually overtaken by environmental shifts. The systems that persist are those that recruit emergent structures satisfying both: creating a distributed network of monitors that use feedback to safely gate-keep the system's efficiency.

\section{The k-Cover and the Closed Loop}
As these emergent monitors are selected and integrated, they naturally form a distributed architecture where every vital resource is monitored by multiple, independent, and heterogeneous checks. 

This creates a self-reinforcing \textbf{closed loop}:
\begin{enumerate}
    \item \textbf{Enhanced Feedback} allows the system to identify ``safe zones'' for growth with high precision.
    \item \textbf{Enhanced Efficiency} allows the system to ``ratchet'' its operational floor to a higher level of performance based on that feedback.
    \item \textbf{Increased Resources} generated from that higher performance are then reinvested into further layers of complexity and monitoring.
\end{enumerate}

This loop explains the tiered nature of biological and social systems—institutions within institutions and loops within loops. Each layer is a historical record of a successful selection for feedback and efficiency. Persistence is not found in a single master-switch, but in the collective influence of these monitors over the system's progress.

\section{A Blueprint for AI Safety}
This perspective shifts the conversation on Artificial Intelligence from external alignment to internal architecture. If the climb toward complexity and internal monitoring is a mathematical necessity for any system that intends to persist, then we should not attempt to bolt safety onto AI as an afterthought.

Instead, we must treat the distributed monitor architecture as the fundamental blueprint for AI design. An AI built on these principles would be an ensemble of monitors where the system’s progress—the ratcheting of its own capabilities—is contingent on the consensus of its monitoring layers. By building AI that mirrors this natural law of persistence, we ensure that as the system becomes more complex, it becomes more—not less—self-regulating. Safety becomes the engine of the system's existence, rather than a constraint upon it.

\section{Conclusion}
The ascent toward complexity is not an accident of history but a structural requirement of existence in a non-equilibrium universe. By recognizing that systems are selected based on their ability to provide both efficiency and feedback, we can better understand the biological world and more safely navigate the creation of artificial ones.

\end{document}