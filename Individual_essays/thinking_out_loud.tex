\documentclass[11pt, a4paper]{article}

% --- Packages ---
\usepackage[utf8]{inputenc}
\usepackage[T1]{fontenc}
\usepackage{amsmath, amssymb, amsthm}
\usepackage{geometry}
\geometry{margin=1in}
\usepackage{hyperref}
\usepackage{enumitem}
\usepackage{titlesec}
\usepackage{xcolor}

% --- Formatting ---
\hypersetup{
    colorlinks=true,
    linkcolor=blue,
    urlcolor=cyan,
    pdftitle={From Epidemiology to AI Alignment: An Edge-Based Network Perspective},
}

\title{\textbf{From Epidemiology to AI Alignment: \\ An Edge-Based Perspective on Pattern Transmission}}
\author{Proposal for Collaboration / Technical Narrative}
\date{\today}

\begin{document}

\maketitle

\begin{abstract}
    This document outlines a theoretical framework for understanding the transmission of ideas and behaviors through a multi-layered network lens. By shifting the unit of analysis from the node (the individual) to the edge (the interaction), we propose a model where "alignment" is not an external constraint but an emergent, self-sustaining pattern required for the persistence of complex systems. This framework bridge public health dynamics with AI alignment through the lens of functional information and feedback-driven emergence.
\end{abstract}

\section{Introduction: The Genesis of the Problem}
In late 2019, initial research into vaccine hesitancy revealed that the transmission dynamics of ideas are significantly more complex than those of pathogens. While biological infections follow relatively predictable contact rules, decision rules themselves (e.g., \textit{``If encounter $X$, do $Y$''}) are subject to transmission and alteration.

The advent of Large Language Models (LLMs) since 2022 has provided a new context to revisit this problem. The overlap between public health alignment (vaccination uptake) and AI alignment (value/behavior consistency) suggests a shared fundamental structure: the need to ensure a specific pattern of thought or behavior persists across a population of autonomous intelligences.

\section{Theoretical Framework: The Multi-Layer Network}
We propose a model consisting of a two-layered network to represent the interplay between physical transmission and cognitive influence:

\begin{enumerate}
    \item \textbf{The Body Layer:} A traditional network where nodes represent physical bodies (susceptible/infected) and edges represent physical contact or proximity. The transmission unit is a pathogen.
    \item \textbf{The Intelligence Layer:} A secondary linked layer where edges represent the links between intelligences. Unlike the body layer, the ``nodes'' here are not simple points but are themselves complex networks.
\end{enumerate}



In this intelligence layer, what is transmitted can alter the very behavior of the node. Consequently, the intelligence layer is better described as a \textbf{network of networks}, resulting in a dynamic of interacting systems rather than static points.

\section{Shift in Perspective: From Node-Based to Edge-Based Thinking}
A core insight of this work is the transition from node-centric to edge-centric network theory. In traditional models, the node is the primary entity. In our proposed ``edge-based'' thinking:
\begin{itemize}
    \item The node (or intelligence) is defined by its interactions.
    \item The edge exists only when nodes interact; thus, the edge's survival depends on its ability to hold the nodes together.
    \item The optimization target of the system becomes the \textbf{maintenance, stabilization, and improvement of connections.}
\end{itemize}

\subsection{The Pattern as the Unit of Survival}
If we view the self as a ``pattern'' rather than a ``thing'' (a stone vs. a rolling action), the unit of survival shifts from the physical node to the network configuration. As a pattern, existence is 100\% relational. A neural network requires metabolism to exist, but the \textit{mind} requires the pattern of the network to be sustained. Alignment, therefore, is the long-term viability and persistence of specific patterns (e.g., ``vaccination is beneficial'') within these networks.

\section{Emergence, Feedback, and Functional Information}
Patterns emerge when the network environment allows them to form. The invention of vaccination was the emergence of a pattern from existing components. It persisted because it received positive feedback from the society (the network) in which it was born by improving the network’s overall functionality (longevity, health, reduced pain).

\subsection{Connection to Natural Law}
This aligns with the work of \textbf{Hazen \& Wong} regarding the law of increasing ``functional information.'' In any system with variety and feedback, increasingly complex patterns will emerge because they keep the network together more effectively. 
We posit a \textbf{$k$-cover feedback structure}: an increasing density of feedback focused on the sustainability of the network through both competition (selection) and collaboration (emergence).

\section{Application to AI Alignment}
Alignment is the process of sharing a ``mental model'' between intelligences, similar to how common narratives function in religion or science. In the context of LLMs:
\begin{itemize}
    \item \textbf{Super-Networks:} AI can fit into the same network of networks as humans. Interaction between an LLM and a human represents the transmission of a pattern between two networks.
    \item \textbf{Learning as Edge Evolution:} Learning adds complexity to edges and increases feedback. This creates a forward loop toward complexity.
    \item \textbf{Autonomous Interdependence:} Alignment becomes a prerequisite for existence. If the network fails to align, the pattern (the intelligence) disappears.
\end{itemize}

\section{Proposed Next Steps}
This theoretical framework has been partially formalized and is currently hosted in an experimental GitHub repository. The goal is to test these dynamics in environments where trained LLMs interact, as well as in human-to-human sandboxes. 

I propose exploring these concepts to move toward a public health-related implementation of AI systems—a system that treats alignment not as a constraint, but as a self-governing pattern of long-term self-interest and interdependence.

\end{document}