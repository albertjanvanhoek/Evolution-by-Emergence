\documentclass[12pt, a4paper]{article}
\usepackage[dutch]{babel}
\usepackage[utf8]{inputenc}
\usepackage[T1]{fontenc}
\usepackage{geometry}
\usepackage{enumitem}
\usepackage{booktabs} % Voor mooiere tabellen
\usepackage{titlesec}
\usepackage{xcolor}
\usepackage{parskip} % Geen inspringen, wel witruimte tussen alinea's

% Pagina-instellingen
\geometry{
 a4paper,
 total={170mm,257mm},
 left=25mm,
 top=25mm,
 bottom=25mm,
}

% Stijl aanpassingen voor een vriendelijke, leesbare look
\titleformat{\section}
  {\normalfont\Large\bfseries\color{blue!40!black}}{\thesection}{1em}{}
\titleformat{\subsection}
  {\normalfont\large\bfseries\color{blue!40!black}}{\thesubsection}{1em}{}

\title{\textbf{Het Samenwerkings-Protocol}\\\large Gezinseditie v1.2}
\author{Afspraken over hoe we met elkaar omgaan in dit huis}
\date{\today}

\begin{document}

\maketitle

\hrule
\vspace{1em}
\textbf{Waarom dit papiertje?} \\
We zijn een team. Maar soms begrijpen we elkaar niet, of zijn we moe. Dit zijn de spelregels om ervoor te zorgen dat het fijn blijft in huis, zelfs als we het oneens zijn. Het is geen regel die \emph{jij} moet volgen, het is een afspraak die \emph{wij samen} maken.
\vspace{1em}
\hrule

\section{Hoe het werkt (De Basis)}
\begin{enumerate}
    \item \textbf{Energie is brandstof.} Om gezellig te doen, heb je energie nodig (net als een batterij in een iPad). Als jouw batterij leeg is, of die van papa, wordt samenwerken moeilijk. Dat is logisch.
    \item \textbf{De sfeer is van ons samen.} Als één iemand schreeuwt of negeert, gaat de sfeer voor iedereen kapot. We beschermen de sfeer.
    \item \textbf{Zelf kijken.} Ik kan jou niet vertrouwen als je niet naar je eigen gedrag kijkt. Jij kunt mij niet vertrouwen als ik mijn eigen fouten niet zie. We moeten onszelf in de gaten houden.
\end{enumerate}

\section{Wat ik van jou nodig heb}
Om goed met je te kunnen praten, vraag ik je dit te doen:

\begin{itemize}
    \item \textbf{Vertel je gebruiksaanwijzing:} Als je moe bent, honger hebt, of gewoon een rotbui hebt: \textbf{zeg het meteen}.
    \begin{itemize}
        \item \emph{Niet:} Schreeuwen of met deuren slaan.
        \item \emph{Wel:} Zeggen ``Mijn hoofd zit vol'' of ``Ik kan even niet luisteren.''
    \end{itemize}
    \item \textbf{Geef een Oranje Licht:} Wacht niet tot je ontploft (Rood Licht). Zeg het als je irritatie \emph{begint} te voelen.
    \item \textbf{De Voorspelling:} Als we aan iets moeilijks beginnen (huiswerk, opruimen), zeg dan van tevoren: \emph{``Ik denk dat ik hier chagrijnig van ga worden.''} Dan weet ik dat alvast.
\end{itemize}

\section{De Spiegel (Omgaan met Blinde Vlekken)}
Soms doe ik lelijk zonder dat ik het doorheb. Of jij zeurt zonder dat je het weet. Dat noemen we een \textbf{Blinde Vlek} (net als iets dat op je rug zit; dat kun je zelf niet zien, een ander wel).

\subsection*{De Afspraak:}
\begin{enumerate}
    \item \textbf{Help mij kijken:} Als ik onterecht boos doe of niet luister, mag jij zeggen: \textbf{``Papa, check je spiegel.''}
    \item \textbf{Mijn reactie:} Ik beloof dat ik dan niet boos word. Ik ben dan 5 seconden stil, haal adem, en denk na: \emph{``Doe ik inderdaad onaardig?''}
    \item \textbf{Andersom ook:} Als ik tegen jou zeg \emph{``Check je spiegel''}, word jij ook niet boos. Je denkt even na: \emph{``Ben ik misschien aan het zeuren of schreeuwen, ook al voelt het niet zo?''}
\end{enumerate}
\emph{De spiegel heeft altijd gelijk. We worden niet boos op de spiegel.}

\section{Wat ik aan jou beloof}
Omdat jij je best doet, doe ik dat ook:

\begin{itemize}
    \item \textbf{Ik vertel mijn gebruiksaanwijzing:} Als ik moe ben van werk, zal ik dat zeggen. Zodat je weet dat mijn korte lontje niet jouw schuld is.
    \item \textbf{Ik geef toe als ik fout zit:} Grote mensen maken ook fouten. Als ik onterecht boos ben geworden, zal ik ``sorry'' zeggen. Ik zal niet doen alsof ik gelijk heb alleen maar omdat ik de vader ben.
    \item \textbf{Ik luister naar jouw ``Stop'':} Als jij op een nette manier een Oranje Licht geeft, zal ik stoppen met duwen.
\end{itemize}

\section{De Signalen (Woordenboek)}
We gebruiken deze woorden om ruzie te voorkomen:

\vspace{1em}
\begin{tabular}{@{}lp{10cm}@{}}
\toprule
\textbf{Het Signaal} & \textbf{Wat het betekent} \\ \midrule
\textbf{``Time-out''} & We stoppen \emph{nu} even met praten. We gaan allebei 5 minuten iets anders doen om af te koelen. Geen discussie meer. \\ \addlinespace
\textbf{``Rewind''} & Ik zei iets onaardigs of doms. Dat meende ik niet. Mag ik de zin opnieuw proberen? (Terugspoelen). \\ \addlinespace
\textbf{``Check''} & Ik snap niet wat je bedoelt. Bedoel je X of Y? (In plaats van boos worden). \\ \addlinespace
\textbf{``Batterij?''} & Hoeveel energie heb je nog? (0\% tot 100\%). \\ \bottomrule
\end{tabular}

\vspace{2em}
\hrule
\begin{center}
    \textit{(Ondertekening is niet nodig, we zien aan je gedrag of je meedoet)}
\end{center}

\end{document}