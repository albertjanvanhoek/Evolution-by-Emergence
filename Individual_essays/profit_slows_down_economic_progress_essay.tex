\documentclass[12pt,a4paper]{article}
\usepackage[utf8]{inputenc}
\usepackage[margin=1in]{geometry}
\usepackage{csquotes}
\usepackage{parskip}
\usepackage{titlesec}

% Section formatting
\titleformat{\section}{\large\bfseries}{\thesection}{1em}{}
\titleformat{\subsection}{\normalsize\bfseries}{\thesubsection}{1em}{}

\title{\textbf{How Profit Can Slow Down Real Economic Growth}\\
\large (When Winning the Game Breaks the Board)}

\author{}
\date{}

\begin{document}

\maketitle

\begin{center}
\textit{An essay about why the forces that create billionaires\\often slow down everyone else's progress---including theirs.}
\end{center}

\vspace{0.5cm}

\noindent You've probably noticed: life-saving drugs exist but cost \$100,000. Brilliant research sits behind paywalls. The people who could cure cancer are driving Uber because they can't make rent. Something in the logic broke.

We're told a simple story about progress:
\begin{itemize}
    \item Profit drives innovation.
    \item Innovation drives growth.
    \item Therefore, more profit $\rightarrow$ more growth.
\end{itemize}

But if you look at the economy not as a set of isolated firms, but as a \textbf{network that learns}, another picture appears:

\begin{quote}
\textbf{Beyond what you need to stay healthy and keep contributing, extra profit that isn't shared back into our common systems doesn't speed us up. It slows us down.}
\end{quote}

\noindent It reduces how fast we, as a whole, can discover, share, and build on new ideas.

This is the central claim of this essay.

\textit{If you're curious about why brilliant solutions exist but don't spread, or why we seem to solve the same problems over and over---this offers one explanation.}

\section{What if growth isn't ``more stuff'', but ``more capability''?}

Let's start by asking a different question:

\begin{quote}
What does it actually mean for an economy to ``grow''?
\end{quote}

Instead of defining growth as \textbf{more production} or \textbf{higher GDP}, define it as:

\begin{quote}
\textbf{Our shared ability to solve problems and create good things, without destroying the conditions we depend on.}
\end{quote}

That includes developing medicines and clean energy, building tools that make people more capable, creating art, culture, and meaning---and doing all this in a way that doesn't burn out people, societies, or ecosystems.

Seen this way, growth is less about \textbf{how much we extract} and more about \textbf{how much we can do together, safely, over longer stretches of time}.

\section{The economy as a learning network---and the ratchet}

Now imagine the economy as a \textbf{huge learning network}: people, teams, companies, labs, and AIs are the points, and the connections between them move ideas, money, trust, and tools.

This network \textit{learns} when three things happen:

\begin{enumerate}
    \item \textbf{Someone creates a new pattern}---an idea, a method, a technology, a better way of organizing or collaborating.
    \item \textbf{That pattern becomes shareable}---others can see it, understand it, afford it, copy it.
    \item \textbf{Others build on it}---adapting it, correcting it, combining it with other ideas.
\end{enumerate}

Every time that loop runs---create $\rightarrow$ share $\rightarrow$ build on---the \textbf{whole} network becomes more capable.

You can think of this as a \textbf{ratchet}:
\begin{itemize}
    \item once a useful pattern is shared and woven into our common systems (infrastructure, knowledge, norms),
    \item it's easier for future generations to start from that level and go further.
\end{itemize}

Think of it like this: every time someone figures out how to purify water cheaply, or write software that automates boring work, or grow food in harsh climates---if that knowledge becomes \textbf{shared infrastructure}, then everyone after them starts from a higher floor. They don't have to re-solve those problems. They can tackle new ones.

But if that knowledge gets locked behind patents, paywalls, or trade secrets? Then a thousand other people waste time re-inventing the same wheel---or never solve the problem at all because they can't afford the \$50,000 access fee.

The \textbf{speed} of our real economic progress is therefore:

\begin{quote}
\textbf{How quickly meaningful new patterns enter our shared systems and get recombined into even better ones.}
\end{quote}

\section{What sets the speed of this ratchet?}

Several things matter for how fast this learning ratchet turns:

\textbf{1. How many people and organizations are above survival mode}

When you're constantly worried about rent, food, or safety, you have little time or energy to create and experiment.

Right now, there's probably someone working three gigs, brilliant enough to spot the flaw in your company's approach or write the paper that would have saved you two years of research. But they're spending their cognitive capacity calculating whether they can afford both groceries \textit{and} electricity this month. The network never gets access to what they could contribute.

\textbf{2. How much capacity they have to contribute}

Education, tools, time to think, mental health, stable institutions.

\textbf{3. How much of what they create is made reusable}

Open knowledge, transparent methods, interoperable standards, affordable access.

Academic researchers joke darkly about using Sci-Hub to access papers they themselves wrote---because their own university can't afford the journal subscriptions. A system where the people who created knowledge can't access it without breaking the law is not optimizing for learning. It's optimizing for extraction.

\textbf{4. How stable the underlying conditions are}

Healthy ecosystems, social trust, reliable institutions, a non-toxic information space.

Every pandemic, every supply chain collapse, every wildfire season that shuts down whole regions---we're spending our collective cognitive capacity on crisis management instead of solving new problems. That's the tax of fragility.

The more people have real capacity, the more they share, and the more stable the foundation, the faster we all move.

That's what \textit{collective} economic growth looks like when you focus on capability rather than just money.

\section{Where profit enters the picture---and where it becomes drag}

Now let's bring profit back in.

There's an important distinction:

\textbf{Profit for viability}---the margin a person or organization needs to:
\begin{itemize}
    \item stay alive and motivated,
    \item invest in better tools and skills,
    \item survive bad years,
    \item keep contributing over time.
\end{itemize}

\textbf{Profit beyond viability}---extra surplus that is:
\begin{itemize}
    \item hoarded,
    \item used mostly for status games,
    \item or used to build walls: lock-in, monopolies, paywalls, or political influence that protects extraction.
\end{itemize}

The second type is where the problem starts.

Why? Because at the level of the network, this ``extra'' profit often shows up as \textbf{non-collaboration}:
\begin{itemize}
    \item charging far above sustainable cost,
    \item enclosing knowledge behind paywalls or restrictive patents,
    \item designing systems that are deliberately incompatible so users can't switch,
    \item lobbying to keep advantages that block others from entering.
\end{itemize}

All of this sends the same signal:

\begin{quote}
``I will use what the network offers me, but I won't proportionally upgrade the network in return.''
\end{quote}

From an individual perspective, that looks smart. From a collective perspective, it's like pulling bricks out of the bridge you're standing on.

\subsection{How this slows real growth}

Once you see the economy as a learning network, the mechanisms become clearer.

\subsubsection*{Fewer people able to contribute}

If large amounts of value are captured and kept at the top, without being used to strengthen shared systems, more people stay close to survival. They work multiple jobs, have little time, no safety net, and limited access to education.

These people might have huge potential, but the network never really ``boots them up'' as full contributors.

Fewer people above survival $\rightarrow$ fewer minds able to create and experiment $\rightarrow$ fewer new patterns.

\subsubsection*{Slower spread and recombination of ideas}

When useful knowledge is locked behind high prices or legal walls, fewer people can \textbf{see} it, fewer can \textbf{afford} it, and fewer can \textbf{tinker} with it.

Think of research papers behind expensive journal paywalls, life-saving drugs priced far above production cost, software that can't interact with other systems.

Every barrier slows the spread of ideas through the network and delays the moment when someone, somewhere, recombines them into something better.

Slower spread $\rightarrow$ slower recombination $\rightarrow$ slower real progress.

\subsubsection*{A more fragile foundation}

Extractive profit models also tend to strip away redundancy (``just-in-time'' everything with no slack), underinvest in maintenance and resilience (``that's someone else's problem''), and ignore environmental and social costs as long as they don't show up on this quarter's balance sheet.

That might look efficient in the short term, but it makes the whole system \textbf{fragile}.

Fragile systems spend more and more time dealing with crises, repairing damage, and managing backlash and distrust.

That's time and energy \textbf{not} spent advancing the frontier.

\section{``But don't we need profit for innovation?''}

We do need profit---up to a point, and in a certain direction.

This argument is \textit{not} ``profit is bad.''

It's:

\begin{quote}
\textbf{Profit helps society grow only when it is used to strengthen shared capability.}
\end{quote}

Some examples of profit that likely \textbf{speeds up} real growth:
\begin{itemize}
    \item investing in open infrastructure (like widely accessible tools, public research, shared standards),
    \item paying people well and giving them time to learn and experiment,
    \item making products cheaper and better over time, widening access,
    \item building resilient supply chains and lowering environmental impact.
\end{itemize}

In these cases, profit is a way of saying:

\begin{quote}
``We created value, and now we're using part of the surplus to help everyone create even more.''
\end{quote}

By contrast, profit that \textbf{slows us down} looks like:
\begin{itemize}
    \item buying back shares instead of investing in useful innovation,
    \item lobbying to block new entrants, keep prices high, or avoid regulation,
    \item keeping critical knowledge secret even when lives or the planet are at stake,
    \item stripping workers and ecosystems to the bone to squeeze out a few more percentage points.
\end{itemize}

In those cases, the message is:

\begin{quote}
``We extracted value from the network and used the surplus mostly to protect our ability to keep extracting.''
\end{quote}

This is the paradox: the strategies that ``win'' under extractive rules---aggressive IP protection, regulatory capture, worker exploitation---are precisely the strategies that degrade the board. And once the board is degraded enough, everyone loses. Even the winners.

That might ``win'' the game in the short term, but it slowly breaks the board everyone is playing on.

\section{What genuine economic progress would aim for}

If we take this seriously, the target for economic policy and corporate strategy shifts.

Instead of asking: ``How do we maximize individual profit?'' we ask:

\begin{quote}
``How do we maximize the \textbf{speed and stability of our shared learning}?''
\end{quote}

That points toward designs that:

\textbf{Lift more people above survival}, so they have time and energy to contribute.

\textbf{Make essential knowledge and tools widely accessible}, so good ideas can spread and recombine quickly.

\textbf{Price things as low as sustainably possible}, not as high as the market will bear, especially for essentials.

\textbf{Tie surplus to contribution back into shared systems}, so that doing very well means you are measurably improving the commons: funding research, strengthening infrastructure, reducing risk and harm for others.

Success, in this view, is less ``How much did \textit{I} get out?'' and more: ``How much did we increase our shared ability to solve problems and live well?''

\section{What this means for you}

If you're building something---a company, a policy, a technology---here's the uncomfortable question this perspective forces:

\begin{quote}
\textbf{Are you using your resources to make the whole system more capable? Or are you quietly mining it?}
\end{quote}

Because here's what the network view reveals: there's no escaping interdependence. The researcher who can't afford to be a researcher, the developer locked out by incompatible systems, the community too stressed to organize---their diminished capacity \textbf{is} your diminished capacity. You just don't see it on your balance sheet.

The most successful entities in any learning network aren't the ones that extract the most. They're the ones that lift the most.

Not because it's morally nice. Because that's how the underlying physics works.

The economy isn't a zero-sum game where your win requires someone else's loss. But it's also not infinitely elastic. It's a learning network with a speed limit, and that speed limit is set by how fast we can develop, share, and recombine good ideas.

Profit beyond what keeps you viable and contributing? It only accelerates growth if it flows back into expanding who can contribute and what they can access.

Otherwise, it's just drag disguised as success.

\section{A note on where this perspective comes from}

This way of thinking doesn't come out of nowhere.

It's part of a broader attempt to understand how \textbf{complex systems}---from ecosystems to brains to economies---evolve over time: how they create and store new patterns, how they share those patterns across the network, and how they maintain stability while still changing and learning.

You don't need to know the whole underlying theory to get the core idea, though:

\begin{quote}
\textbf{Real economic growth is a learning process in a shared system. Profit speeds that process up only when it is used to strengthen the system itself. Beyond that, profit becomes drag.}
\end{quote}

\section*{In one sentence}

If you want the bumper-sticker version:

\begin{quote}
\textbf{Profit accelerates progress only when it upgrades the commons. Beyond that, it's drag.}
\end{quote}

That's the uncomfortable truth hiding behind the comforting story that ``profit always drives progress.''

Sometimes it does.

But often, especially once basic needs and viability are covered, it quietly does the opposite.

\end{document}