\documentclass[11pt, a4paper]{article}
\usepackage[utf8]{inputenc}
\usepackage[T1]{fontenc}
\usepackage{amsmath}
\usepackage{amssymb}
\usepackage{geometry}
\geometry{margin=1in}
\usepackage{parskip} % Adds space between paragraphs for readability

\begin{document}

\section*{Evolution as Selection under Viability Constraints}

We conceptualize evolution as selection under the constraints of existence. Central to this framework is the \textit{viability set} ($K$): the set of states within which a system can persist, and outside of which failure constitutes an \textbf{absorbing state}. This set encompasses both physical thresholds (such as energy and integrity) and relational prerequisites (such as connectivity and trust), defined across multiple scales—from individual subsystems to the network as a whole.

Within this dynamic, variation continuously emerges through collisions, recombination, and the reconfiguration of existing structures. Because survival probabilities depend on relations—specifically the fluxes of energy, information, resources, and regulation—survival is fundamentally a \textbf{network property}: entities typically exist only as components of supporting configurations. Retention mechanisms (memory or replication) render these configurations repeatable, shifting the statistical distribution of surviving systems toward structures that are more robust within ($K$). Formally, we consider trajectories conditioned on $x_t \in K$ for all $t$; through this conditioning, the preservation of structure compatible with viability is statistically privileged. This process induces selection pressure toward functional organization—such as modularity, redundancy, and feedback loops—and allows diversity to increase as long as it enhances viability.

The web accumulates this functional organization only as long as there is a sufficient \textbf{free energy} budget and net throughput to sustain the costs of retention, maintenance, and error correction. Local order requires dissipation (entropy production elsewhere); when usable gradients flatten, maintenance costs become relatively prohibitive, and organization degrades.

With the emergence of intelligence, a new factor appears: the network can model its own viability conditions and actively pursue strategies to protect access to free energy, buffers, and recovery capacity. Success, however, is not guaranteed. It depends on the extent to which local incentives and power dynamics (e.g., capture, races, Goodhart's law) align with global viability. Securing the boundary conditions for persistence thus becomes absolute: goals that structurally run counter to the physical basis of the system are, in the long run, inherently self-undermining.

\end{document}