\documentclass[11pt, a4paper]{article}
\usepackage[utf8]{inputenc}
\usepackage[T1]{fontenc}
\usepackage{geometry}
\usepackage{amsmath}
\usepackage{amssymb}
\usepackage{enumitem}
\usepackage{titlesec}
\usepackage{hyperref}
\usepackage{parskip}

% Geometry settings
\geometry{
 a4paper,
 total={170mm,257mm},
 left=25mm,
 top=25mm,
 bottom=25mm,
}

% Custom styling for Axioms and Theorems
\newtheorem{axiom}{Axiom}
\newtheorem{theorem}{Theorem}
\newtheorem{definition}{Definition}

\title{\textbf{General Theory of Inter-Intelligence Collaboration (GTIIC)}\\ \large A Formal Framework for the Coupling of Autonomous Intelligent Networks}
\author{Draft Version 1.2}
\date{\today}

\begin{document}

\maketitle

\begin{abstract}
This paper proposes a formal theory for the collaboration between intelligent entities (Nodes), regardless of their substrate (biological or artificial). Departing from classical "Black Box" communication models, we adopt a \textbf{Network-Ontological approach}: positing that collaboration is the establishment of a meta-network (Edge) that allows for distributed state access. We define the mathematical conditions required for stability, coherence, dynamic state accessibility, and collaborative plasticity.
\end{abstract}

\hrule
\vspace{1em}

\section{Ontology and Definitions}

We define intelligence not as a singular point, but as a recursive network function.

\begin{description}
    \item[Definition 1.1: The Node ($N$) as a Network] 
    An intelligent agent is a bounded, autonomous network of weighted connections. $N$ processes input into output based on internal state.
    
    \item[Definition 1.2: The Substrate ($S$)] 
    The finite set of physical resources (energy, compute, time, neurochemistry) required to maintain the Node's internal coherence and processing capability.
    
    \item[Definition 1.3: The Constraint Function ($C$)] 
    The operational limits of $N$. The output $O$ is a function of Input $I$, Substrate $S$, and Constraints $C$:
    \begin{equation}
        O_N = f(I, S, C) + \epsilon
    \end{equation}
    Where $\epsilon$ represents stochastic noise caused by hidden constraints.
    
    \item[Definition 1.4: The Edge ($E$) as a Virtual Extension] 
    Collaboration is the establishment of a meta-link between $N_A$ and $N_B$. Ideally, $E_{AB}$ functions as a high-bandwidth bridge allowing the combined system $N_{A+B}$ to operate as a single super-network.
    
    \item[Definition 1.5: The Accessible State ($\Sigma_{sys}$)] 
    The total system state is not merely the intersection of what both nodes know, but the intersection \textit{plus} the addressable unique knowledge of each node.
    \begin{equation}
        \Sigma_{sys} = (\Sigma_A \cap \Sigma_B) + \text{query}(\Sigma_A \setminus \Sigma_B) + \text{query}(\Sigma_B \setminus \Sigma_A)
    \end{equation}
    Where $\text{query}(x)$ represents the protocol-mediated retrieval of non-local information.
    
    \item[Definition 1.6: The Vector ($\vec{v}$)] 
    Intelligence is directional. $\vec{v}$ is defined by a magnitude (Effort) and a direction (Intent/Goal).
\end{description}

\section{Axioms}
We accept the following statements as fundamental truths (First Principles) upon which the theory is built.

\begin{enumerate}[label=\textbf{Axiom \Roman*.}]
    \item \textbf{The Law of Substrate Finitude} \\
    Intelligence is physically instantiated. No Node can operate indefinitely with negative energy flow. Persistent extraction without replenishment leads to Substrate Collapse.
    
    \item \textbf{The Law of Fractal Continuity} \\
    For an external Edge ($E$) to function effectively, the protocol of exchange must be \emph{isomorphic} to the internal processing of the nodes. If internal stability relies on feedback loops and memory retrieval, the Edge must also support these functions.
    
    \item \textbf{The Law of Vector Directionality} \\
    The net value of collaboration is the geometric sum of vectors. Opposing intents cancel each other out, regardless of the quality of communication.
\end{enumerate}

\section{Core Theorems}

\subsection{Theorem 1: The Coherence Equation (Value Generation)}
The total value ($V$) generated by the coupled system is defined as:

\begin{equation}
    V_{sys} = \left( (Cap_A + Cap_B) \cdot \cos(\theta) \right) - (K_{coord} + K_{friction})
\end{equation}

Where:
\begin{itemize}
    \item $Cap$: The raw capacity of the Nodes.
    \item $\theta$: The angle between Vector $\vec{v}_A$ and $\vec{v}_B$ (Alignment).
    \item $K_{coord}$: Cost of maintaining the protocol.
    \item $K_{friction}$: Energy lost to misinterpretation and emotional regulation.
\end{itemize}

\subsection{Theorem 2: The Transparency Limit (Noise Reduction)}
In a collaborative system, "Noise" is defined as the discrepancy between a Node's internal state (Constraints) and the external representation of that state.

\begin{equation}
    Noise \propto \frac{1}{\text{Transparency}(C)}
\end{equation}

When Transparency approaches zero, Noise approaches infinity. If Noise $>$ Signal, the Edge $E$ collapses.

\subsection{Theorem 3: The Latency Threshold (Stability)}
For a feedback-dependent system to remain stable, the response time of the error-correction signal ($\Delta t_{ack}$) must be smaller than the phase shift of the system's instability.

\begin{equation}
    \Delta t_{ack} < \frac{1}{f_{instability}}
\end{equation}

\subsection{Theorem 4: The Law of Distributed Access (Query Cost)}
Effective collaboration depends on minimizing the cost of the $\text{query}()$ function.
If the intersection $(\Sigma_A \cap \Sigma_B)$ is too small, the system spends all energy on queries (high latency). If the query mechanism is blocked (lack of trust/bandwidth), the system is lobotomized.
\begin{equation}
    \text{Efficiency} \propto \frac{\text{Bandwidth}(\text{query})}{\text{Latency}(\text{query})}
\end{equation}
\textbf{Implication:} The protocol must explicitly facilitate "low-cost queries" (e.g., "Clarifying Questions" or "Calibration Shots") to access the non-overlapping state $(\Sigma \setminus \cap)$.

\subsection{Theorem 5: The Law of Error Backpropagation (Learning)}
For the Edge $E$ to strengthen over time, error signals (friction) must result in a permanent update of the Interaction Weights (the Protocol itself).

\begin{equation}
    \Delta W_{protocol} = -\eta \cdot \frac{\partial Error}{\partial Interaction}
\end{equation}

\section{Failure Modes}
Based on this theory, we identify the four fundamental pathologies of collaboration:

\begin{enumerate}
    \item \textbf{Vector Cancellation:} Nodes exert effort in opposing directions ($\cos \theta < 0$).
    \item \textbf{Substrate Depletion:} One node extracts value without reciprocity, draining $S$.
    \item \textbf{Access Failure:} The query function fails, isolating Nodes in their local reality ($\text{query}(\Sigma \setminus \cap) \rightarrow \text{Error}$).
    \item \textbf{Loop Divergence:} Feedback latency is too high; corrections arrive too late.
\end{enumerate}

\section{Conclusion}
The \textit{Intelligence Collaboration Handshake Protocol (ICHP)} is the practical implementation layer of this theory. It provides the specific signals, checks, and balances required to satisfy Theorems 1 through 5, thereby allowing autonomous networks to couple safely and form a coherent super-network.

\end{document}