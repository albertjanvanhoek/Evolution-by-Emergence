
\documentclass[11pt,a4paper]{article}
\usepackage[utf8]{inputenc}
\usepackage[T1]{fontenc}
\usepackage{lmodern}
\usepackage[margin=1in]{geometry}
\usepackage{setspace}
\usepackage{hyperref}
\hypersetup{colorlinks=true,linkcolor=blue,citecolor=blue,urlcolor=blue}

\title{Pricing Vaccines for Collective Regenerative Capacity}
\author{}
\date{\today}

\begin{document}
\maketitle
\doublespacing

\begin{abstract}
Vaccine pricing is often framed as a technical extension of cost-effectiveness analysis (CEA) or ``value-based'' models. We argue this misses the central question: \emph{what price maximizes society's ability to keep getting better over time?} We call that ability \textbf{Collective Regenerative Capacity (CRC)}. The paper develops a clear, non-technical logic: why CRC matters, how security and planning horizons connect vaccination to CRC, why CEA is essential for deciding \emph{what} to fund but not sufficient for deciding \emph{how} to price, and what high-level principles should guide pricing to achieve fast, broad coverage without being naive about incentives or sustainability.
\end{abstract}

\section{Introducing CRC: A System's Ability to Keep Improving}
\textbf{Definition.} CRC is the shared capacity of a society to generate better outcomes for everyone, and to do so faster over time. It has two aspects: the \emph{quality} of the base we stand on today (health, knowledge, institutions, environment), and the \emph{speed} with which that base improves.

\textbf{Ecology as intuition.} In a healthy forest, soils, water cycles, and biodiversity support continual renewal. Over-extraction damages both the current state and the forest's ability to recover. Investment---like restoring soil and protecting pollinators---preserves the present and accelerates future growth. Human societies work similarly: public health, education, research, infrastructure, and trust-bearing institutions are the ``soil'' from which progress grows.

\textbf{Why vaccines are emblematic.} Immunization prevents catastrophic shocks, protects health services from overload, sustains productivity, and preserves confidence. In short: vaccination protects the \emph{base} and the \emph{speed} of improvement. Pricing that expands timely coverage strengthens CRC; pricing that rations access weakens it.

\section{Security Extends Planning Horizons}
People make different choices depending on how secure they feel. When basic risks are buffered (illness, income loss, violence), it becomes rational to think and invest over longer horizons. When risks are unbuffered, short-term decisions are rational---even if they are collectively harmful. Vaccines directly increase security by reducing the chance that a single health event derails a household or a clinic. As security rises, planning horizons lengthen, and long-term investments become the norm. This is the engine by which high coverage in turn accelerates CRC.

\section{Right Tools, Right Jobs: CEA vs.\ Pricing}
CEA is indispensable for \emph{allocation} questions: given a budget, which interventions deliver the most health for the money? Pricing is a different problem. Pricing sets the affordability at the point of delivery---and therefore the level and speed of coverage. Because coverage shapes security and planning horizons, \emph{pricing} is a lever on CRC. The practical takeaway is simple: use CEA to decide \emph{what to buy}; use CRC logic to decide \emph{how to price} so that coverage is fast and broad.

\section{Guiding Principles for Vaccine Pricing (Without the Math)}
We do not need equations to set direction. A CRC-aware pricing approach should be guided by five plain-language principles:
\begin{enumerate}
    \item \textbf{Maximize timely coverage.} The public goal is the fastest possible protection for the broadest possible population, especially where price sensitivity is steep.
    \item \textbf{Recover real costs transparently.} ``At cost'' should include production, distribution systems, and a clear path to recover legitimate R\&D, with public reporting and sunset clauses once recovered.
    \item \textbf{Price where it moves coverage the most.} Tier prices by ability-to-pay and observed coverage response, with the steepest reductions where price stands most in the way.
    \item \textbf{Lower coordination friction.} Favor open technical standards, predictable long-term volumes, and diversified manufacturing to reduce delays and supply shocks.
    \item \textbf{Align rewards with long-run performance.} Use tools like advance commitments, delivery benchmarks, and quality safeguards that reward reliable supply and penalize avoidable delays or failures.
\end{enumerate}
These principles are compatible with market-shaping instruments many agencies already use. The shift is \emph{purpose}: from maximizing private willingness-to-pay to maximizing coverage that strengthens CRC while ensuring credible, audited cost recovery.

\section{The Horizon Threshold: Goal, Measurement, and Governance}
\subsection*{Goal}
Keep population planning horizons \emph{above} a security threshold where long-term choices (prevention, education, maintenance) become the default. Pricing and coverage are instruments to hold the system on that side of the line.

\subsection*{What We Can Measure (directly or by proxy)}
\textbf{Direct (survey):}
\begin{itemize}\setlength\itemsep{0.2em}
\item ``How far ahead do you usually plan important financial/health decisions?'' (months/years)
\item ``Would you accept a guaranteed 10-year return over a risky 1-year gain?'' (trade-off item)
\item ``How confident are you that a major illness would be financially manageable?'' (security scale)
\end{itemize}

\textbf{Behavioral/administrative proxies:}
\begin{itemize}\setlength\itemsep{0.2em}
\item Preventive care adherence (e.g., on-time childhood vaccination, cancer screening uptake)
\item Retirement/pension contributions; durable savings penetration; emergency-buffer prevalence
\item Tenure stability (housing, employment); share of fixed-rate long-term mortgages
\item School continuation to planned grade; uptake of multi-year training/apprenticeships
\end{itemize}

\textbf{System-level proxies:}
\begin{itemize}\setlength\itemsep{0.2em}
\item Catastrophic health expenditure rate (decreasing suggests longer horizons)
\item Stockout days for essential vaccines/medicines (fewer shocks)
\item High-cost debt and petty crime tied to desperation extraction (declining with security)
\end{itemize}

\subsection*{Finding the Threshold (pragmatic)}
Look for \emph{inflection}: where small gains in security (insurance coverage, timely vaccination, income stability) yield big jumps in long-horizon behaviors. In practice: compare groups around eligibility cut-offs; pilot security boosts (insurance plus emergency buffer) and track behavior change; and watch for nonlinear responses in the proxies above.

\subsection*{Horizon-Aware Governance}
\textbf{Guardrails to stay above the threshold:}
\begin{itemize}\setlength\itemsep{0.2em}
\item \emph{Coverage first:} price vaccines to remove affordability barriers where price sensitivity is steep.
\item \emph{Protect security buffers:} maintain primary care, immunization, and income stabilizers during downturns.
\item \emph{Reduce friction:} pooled procurement, open tech transfer, diversified manufacturing to avoid supply shocks.
\item \emph{Align rewards with reliability:} advance commitments tied to delivery timeliness and quality.
\end{itemize}

\textbf{Early-warning triggers:} if catastrophic spending, stockouts, or preventive adherence move the wrong way, automatically relax prices/fees and deploy surge supply to restore security and planning horizons.

\section{What This Explains}
This lens clarifies why some places accelerate while others stall. Where safety nets and accessible vaccination extend planning horizons, people and institutions invest---and the system compounds. Where insecurity and high prices ration coverage, short-term extraction becomes rational, CRC erodes, and progress slows. The difference is not moral character; it is system design.

\section{Conclusion}
Vaccine pricing is not a technical afterthought to CEA. It is a system choice about who gets protected, how fast, and whether society's regenerative capacity strengthens or frays. A CRC-aware approach---maximize timely coverage, recover real costs transparently, reduce friction, and align rewards with reliability---is both principled and practical. The details of contracts and formulas can follow. The direction of travel should be clear.

\end{document}
