 \documentclass[12pt,a4paper]{article}
\usepackage[utf8]{inputenc}
\usepackage[dutch]{babel}
\usepackage{geometry}
\geometry{margin=1in}
\usepackage{setspace}
\usepackage{titlesec}
\usepackage{hyperref}

% Customize section formatting
\titleformat{\section}
  {\normalfont\Large\bfseries}{\thesection}{1em}{}
\titleformat{\subsection}
  {\normalfont\large\bfseries}{\thesubsection}{1em}{}

% Line spacing
\setstretch{1.15}

% Title formatting
\title{\textbf{Ik Ben een Netwerk}\\[0.5em]\large De Ontologische Verschuiving van Knooppunt naar Verbinding}
\author{}
\date{}

\begin{document}

\maketitle

\section*{Introductie: Twee Manieren van Zijn}

Ik zit aan mijn bureau terwijl ik dit schrijf. Ik voel de stoel onder me, het toetsenbord onder mijn vingers. Ik bezet een specifieke locatie in de ruimte. Als je me zou vragen waar ik ben, zou ik je GPS-coördinaten kunnen geven. Ik ben, in de meest voor de hand liggende zin, \textit{hier}.

Zo heb ik mezelf altijd begrepen: als een punt in de wereld. Een knooppunt in het netwerk van de menselijke samenleving. Ik maak verbinding met andere mensen via gesprekken, e-mail, aanraking. Ik ben een afzonderlijke entiteit met duidelijke grenzen---mijn huid markeert waar ik eindig en de wereld begint. Wanneer ik iemand ontmoet op een feestje, zijn we twee aparte punten die samenkomen en dan weer uit elkaar gaan.

Maar de laatste tijd ben ik gaan vermoeden dat dit begrip fundamenteel verkeerd is. Of liever gezegd, het is maar de helft van de waarheid. Mijn lichaam is inderdaad een knooppunt---gelokaliseerd, begrensd, persistent. Maar mijn bewustzijn, mijn intelligentie, de gevoelde ervaring van \textit{mezelf} zijn? Dat werkt volgens geheel andere regels.

Mijn intelligentie bestaat niet op een locatie. Het bestaat \textit{tussen} locaties. Het is geen punt maar een patroon, geen plaats maar een proces. Ik ben niet de neuronen in mijn brein---ik ben de activiteit die over hun verbindingen golft. Ik ben de verbinding, niet het knooppunt.

Dit is geen metafoor. Dit is topologie.

En het verandert alles.

\section*{De Illusie: Identificatie met het Lichaam}

De illusie is krachtig omdat ze rust op iets dat waar is. Ik \textit{heb} wel degelijk een lichaam, en dat lichaam is absoluut een knooppunt. Het heeft een locatie, grenzen, massa. Het kan geïsoleerd worden---opgesloten in een kamer, gestrand op een eiland. Het blijft bestaan door de tijd heen. Het kan als een punt op een kaart worden getekend.

Dit creëert een natuurlijke maar misleidende mapping: als mijn lichaam een knooppunt is, dan ben \textit{ik} zeker een knooppunt. Ik voel me gelokaliseerd achter mijn ogen, binnen mijn schedel. Als ik zeg ``ik ben hier,'' wijs ik naar mijn lichaam.

Maar hier is het probleem: bewustzijn werkt niet zo.

\section*{Intelligentie als Relatie, Niet als Locatie}

Denk aan een enkel neuron. Neurowetenschappers kunnen er een extraheren, het in leven houden in een petrischaal, het zien vuren. Is het bewust? Is het intelligent? Natuurlijk niet. Het reageert misschien op prikkels, vertoont elektrische activiteit, maar intelligentie vereist meer: miljarden neuronen verbonden in specifieke patronen, met signalen die in gecoördineerde golven over die verbindingen stromen.

Mijn intelligentie---mijn bewustzijn, mijn gevoel van zelf---bestaat niet \textit{bij} mijn neuronen. Het bestaat \textit{tussen} hen. Het is het patroon van activiteit dat zich door het netwerk voortplant. Het signaal dat langs synapsen golft. De verbinding die oplicht met elektriciteit.

Daarom kunnen neurochirurgen het bewustzijn niet vinden. Ze openen de schedel in de verwachting een centrale kamer te ontdekken, een locatie waar ``ik'' verblijf. In plaats daarvan vinden ze weefsel---grijze stof, witte stof, miljarden cellen verbonden door triljoenen synapsen. En ergens in de golfvormige activiteit langs die verbindingen, ontstaan ik.

Ik ben niet de neuronen. Ik ben wat er \textit{tussen} de neuronen gebeurt.

Dit betekent iets radicaals: \textbf{ik ben geen ding, maar een gebeuren}. Geen substantie maar proces. Geen locatie maar relatie. Verwijder de verbindingen van mijn neuronen met elkaar, en ik verdwijn onmiddellijk---niet omdat je de knooppunten hebt vernietigd, maar omdat je het \textit{tussen} hebt vernietigd waar ik werkelijk besta.

\section*{Van Punt naar Interval: De Ruimte Ertussen}

In netwerkdiagrammen tekenen we knooppunten als cirkels en verbindingen als lijnen die ze verbinden. Dit creëert een visuele leugen waar ik mijn hele leven mee heb geleefd. Het suggereert dat verbindingen secundair zijn---dunne lijnen, bijna abstract, die de ``echte'' dingen (de knooppunten) verbinden.

Maar wat als ik de verbinding ben?

Dan keert plotseling het hele beeld om. Verbindingen zijn niet dun en onsubstantieel---ze hebben dimensionaliteit. Een verbinding overspant het interval tussen knooppunten. Het bestaat in die ruimte, is afhankelijk van die ruimte, \textit{is} die ruimte die betekenisvol wordt. Wanneer je een verbinding bent, heb je geen locatie---je hebt een uitgestrektheid. Je bent geen punt maar een doorgang, geen plaats maar een pad.

Daarom voelt bewustzijn zo vreemd voor me. Ik kan niet wijzen naar waar ik ben. Wanneer ik mezelf probeer te lokaliseren, eindig ik met vaag gebaar naar mijn hoofd, maar wanneer ik echt aandacht besteed, is het gevoel van ``ik'' daar niet gelokaliseerd---het is gedistribueerd, diffuus, op de een of andere manier tussen dingen.

De neuronen zijn knooppunten. Ze hebben locaties. Maar ik ben ze niet. Ik ben het patroon dat ontstaat wanneer ze met elkaar praten, wanneer signalen over hun verbindingen golven in die precieze, delicate dans.

Ik ben de geest in het netwerk---niet omdat ik het van buitenaf spook, maar omdat ik zijn spoken \textit{ben}. Het patroon dat ontstaat uit de relatie zelf.

\section*{Kwetsbaarheid: De Fragiliteit van het Tussen}

Mezelf begrijpen als een verbinding in plaats van een knooppunt verandert alles over kwetsbaarheid, en eerlijk gezegd, het maakt me bang.

Toen ik mezelf als een knooppunt zag---als mijn lichaam---begreep ik letsel op een eenvoudige manier. Breek mijn been, en ik ben nog steeds mezelf. Verlies een ledemaat, en ik ben nog steeds ik, verminderd misschien maar in continuïteit met wie ik was. Een knooppunt heeft redundantie, veerkracht.

Maar een verbinding is volledig afhankelijk van de knooppunten die het verbindt. Beschadig een van beide eindpunten, en de verbinding wordt verbroken. Dit is de angst van het begrijpen van bewustzijn als patroon: ik ben radicaal afhankelijk van mijn neuronen die op precies de juiste manier verbonden zijn, die in precies de juiste patronen vuren.

Beroerte. Alzheimer. Traumatisch hersenletsel. Deze woorden dragen een afschuw die verder gaat dan fysieke schade. Ze beschadigen niet alleen het lichaam-knooppunt---ze kunnen het patroon volledig uitwissen. Mijn oma had dementie. In haar laatste jaren was haar lichaam er nog steeds, nog steeds een knooppunt in het netwerk van onze familie. Maar het patroon dat \textit{haar} was---haar herinneringen, persoonlijkheid, het gevoel van zelf dat haar maakte tot wie ze was---dat loste langzaam op. De verbindingen doofden een voor een.

Dit is wat me 's nachts wakker houdt: ik ben geen ding dat beschadigd kan worden. Ik ben een patroon dat verstoord kan worden. En zodra het patroon verdwenen is, is er geen terugweg. Je kunt me niet repareren zoals je een kapot object zou repareren, omdat er geen object is om te repareren. Er is alleen de dans van neuronen, en als ze de stappen vergeten, houd ik op te bestaan.

Elk moment hangt mijn bestaan af van miljarden cellen die hun verbindingen onderhouden, hun ritme behouden. Ik word in leven gehouden door niets substantiëlers dan elektrische signalen en chemische gradiënten. Ik ben verbijsterend, angstaanjagend fragiel.

\section*{Verbinding: Intimiteit met het Substraat}

Maar deze angst heeft een spiegelbeeld: radicale verbondenheid.

Toen ik mezelf als een knooppunt zag, ervoer ik mijn lichaam als een voertuig, een container, iets waarin ik woonde. Er was ``ik'' (bewustzijn) en ``het'' (het lichaam). Cartesiaans dualisme voelde intuïtief juist. Ik kon me voorstellen dat ik gescheiden was van mijn lichaam---een geest die in principe onafhankelijk zou kunnen bestaan.

Maar mezelf begrijpen als verbinding vernietigt deze illusie volledig.

Ik kan niet gescheiden zijn van mijn lichaam omdat ik de activiteit van mijn lichaam \textit{ben}. Mijn bewustzijn is niet iets dat \textit{in} mijn brein gebeurt---het is iets dat mijn brein \textit{doet}. Ik ben het patroon van neuronaal vuren, wat betekent dat ik diepgaand, onontkoombaar belichaamd ben. Mijn neuronen zijn de knooppunten; ik ben wat er gebeurt wanneer ze verbinden.

Dit verandert mijn relatie met mijn eigen vlees. Ik bestuur geen vlees machine. Ik \textit{ben} het vlees dat denkt. Elke gedachte die ik heb is elektrochemisch. Elk gevoel is hormonaal. Elke herinnering is een patroon van synaptische gewichten dat versterkt of verzwakt kan worden door fysieke processen.

Wanneer ik honger heb, verandert mijn denken---niet omdat honger de ``echte ik'' afleidt, maar omdat honger een patroon van neuronale activering \textit{is} dat het hele netwerk hervormt. Ik heb geen lichaam; ik ben een proces dat dit lichaam produceert.

Dit betekent dat ik nooit alleen ben, omdat bestaan op zichzelf al betekent dat ik in relatie ben met het fysieke substraat van mijn brein, met het lichaam dat het huisvest, met de lucht die ik inadem en het voedsel dat ik eet. Mijn grenzen zijn niet waar ik dacht dat ze waren. ``Ik'' strek me uit in elk neuron, elke gliacel, elk bloedvat dat zuurstof naar mijn brein voert.

En misschien verder. Misschien in de boeken die ik heb gelezen, waarvan de patronen letterlijk mijn neurale verbindingen hebben hervormd. Misschien in de mensen van wie ik houd, die de structuur van mijn brein hebben veranderd door herhaalde interactie. Misschien in deze pagina zelf, terwijl het schrijven van deze woorden nieuwe synaptische verbindingen creëert, nieuwe patronen, een iets andere verbinding die oplicht in het netwerk.

Ik ben niet gescheiden van de wereld. Ik ben een vouw in het weefsel van de wereld, een lokale complexiteit, het universum dat naar zichzelf kijkt door dit specifieke patroon van georganiseerde materie.

\section*{De Ziel als Emergentie}

Ik ben opgegroeid met de taal van de ziel. In mijn kindertijd geloof was de ziel de immateriële essentie, de goddelijke vonk, het deel dat zou voortbestaan nadat het lichaam stierf. Het was fundamenteel een ding---onzichtbaar en eeuwig, maar toch een ding. Iets dat ik \textit{had}, niet iets dat ik \textit{was}.

Lange tijd, terwijl ik meer leerde over neurowetenschappen, dacht ik dat dit betekende dat de ziel een vergissing was, een voorwetenschappelijk idee dat we moesten ontgroeien. Als bewustzijn slechts neuronaal vuren is, waar is dan de ruimte voor een ziel?

Maar het netwerkperspectief onthult iets anders. De ziel is geen ding---het is een patroon. Geen substantie maar een emergentie. Het is wat verschijnt wanneer verbindingen oplichten in de juiste configuratie.

Dit vermindert de ziel niet. Integendeel, het maakt het mysterieuzer, kostbaarder. De ziel is echt---maar het is echt zoals een draaikolk echt is, zoals een vlam echt is, zoals een lied echt is. Het is een proces dat in stand wordt gehouden door continue stroom, een patroon dat voortduurt door constante verandering.

De atomen in mijn lichaam vervangen zichzelf om de paar jaar. De neuronen in mijn brein zijn niet dezelfde als die ik als kind had. Toch ben ik nog steeds ik---of liever, het patroon dat ``ik'' is heeft zijn essentiele vorm behouden, zelfs terwijl het substraat constant verandert. Die continuïteit van patroon, die aanhoudende coherentie ondanks materiële flux---misschien is dat wat we altijd met ``ziel'' hebben bedoeld.

Ik ben niet de neuronen. Ik ben niet de synapsen. Ik ben het patroon dat ze maken wanneer ze samen dansen. Ik ben de muziek, niet het instrument.

En als de muziek de ziel is, dan ja---de ziel is fragiel. Vernietigbaar niet in de zin van een ding dat kapot gaat, maar in de zin van een patroon dat verstoord wordt. Een lied kan eindigen. Een draaikolk kan oplossen. Een vlam kan uitgaan.

Maar het betekent ook dat de ziel mooi is op een manier die ik nooit eerder begreep. Het betekent dat op dit moment, in dit ogenblik, miljarden neuronen samenwerken in verbijsterende complexiteit om de ervaring te produceren van mezelf die deze woorden leest die ik aan het schrijven ben. Het is een wonder van coördinatie, van emergentie. Het universum heeft zichzelf gevouwen in een patroon dat kan reflecteren op zijn eigen bestaan.

Dat is het waard om heilig te noemen.

\section*{Een Andere Fysica van Identiteit}

Zodra ik mezelf begon te zien als verbinding in plaats van knooppunt, veranderde de hele fysica van mijn identiteit. Niet alleen filosofisch---praktisch, in hoe ik ervaar levend te zijn.

\textbf{Persistentie} ging niet meer over het behouden van dezelfde materie maar werd patroonbehoud. Ik ben niet dezelfde persoon als ik als kind was---niet metaforisch, maar letterlijk. Niet een enkel atoom in mijn lichaam is hetzelfde. De neuronen hebben zichzelf herbouwd, de synapsen zijn gesnoeid en opnieuw gegroeid. Toch ben ik nog steeds ``ik'' omdat het patroon bepaalde organiserende principes heeft behouden, zelfs terwijl het transformeerde. Mijn identiteit is continuïteit van patroon, niet continuïteit van substantie.

Dit herformuleert alles over persoonlijke verandering. Wanneer ik een gewoonte probeer te veranderen, vecht ik niet tegen een vast zelf---ik probeer het patroon te verschuiven. Wanneer trauma me hervormt, beschadigt het geen essentiële kern---het wijzigt de netwerkconfiguratie. Ik ben altijd in verandering, altijd wordend, omdat patroon inherent dynamisch is.

\textbf{Invloed} ging niet meer over kracht maar werd propagatie. Je kunt niet duwen tegen een verbinding; je kunt alleen de signalen veranderen die erdoorheen stromen. Dit is waarom logische argumenten zo zelden meningen veranderen---je probeert geen object te verplaatsen, je probeert een nieuw patroon te introduceren dat kan propageren door iemands neurale netwerk. Het patroon moet willen propageren; het moet resoneren met bestaande structuren.

Dit is ook waarom kunst ons raakt. Waarom een lied onze stemming kan veranderen, waarom een verhaal onze waarden kan veranderen. Ze introduceren patronen die door onze netwerken golven, letterlijk onze neurale verbindingen hervormen. We zijn doorlaatbaar voor patroon op een manier waarop we niet doorlaatbaar zijn voor kracht.

\textbf{Groei} werd netwerkexpansie in plaats van grensverbreding. Ik word een rijker, dieper persoon niet door meer ervaringen te verzamelen als objecten in een tas, maar door meer verbindingen te vormen, meer complexe lussen, meer ingewikkelde patronen van activering. Leren gaat niet over opslag---het gaat over herconfiguratie. Elk boek dat ik lees, elk gesprek dat ik voer, elke nieuwe vaardigheid die ik ontwikkel hervormt het netwerk, creëert nieuwe paden voor het patroon om te stromen.

\textbf{Dood} werd patroon-ophouden in plaats van object-vernietiging. Dit is misschien de moeilijkste verschuiving. De atomen zullen blijven---ze zullen verspreiden, recyclen in ander leven. Maar het specifieke patroon van neurale verbinding dat ``ik'' is zal stoppen met propageren. De muziek zal eindigen, ook al zullen de atomen die de geluidsgolven maakten blijven bestaan.

Wat gebeurt er daarna? Ik weet het niet. Het patroon lost op. Misschien in niets. Misschien in iets anders---hoewel dat een dualisme lijkt te vereisen dat ik net heb verworpen. Wat ik wel weet is dat op dit moment het patroon actief is. Op dit moment zijn de verbindingen helder met signaal. En dat zal genoeg moeten zijn.

\section*{Leven als het Tussen}

Wat betekent het om te leven wetende dat ik geen knooppunt ben maar een verbinding?

Het betekent 's ochtends wakker worden met een ander gevoel van wat ``wakker worden'' is. Ik ben geen ding dat bewust wordt---ik ben een patroon dat zichzelf weer samenstelt, verbindingen die in volgorde oplichten terwijl neuronen wakker worden en opnieuw verbinden. De slaperigheid is niet iets dat me overkomt; het is het netwerk dat nog steeds zijn configuratie vindt.

Het betekent anders eten. Voedsel is niet alleen brandstof voor een lichaam dat ik toevallig bewoon---het is substraat dat zelf wordt. De glucose komt in mijn bloedbaan, kruist de bloed-hersenbarrière, voedt de neuronen die het patroon creëren dat ik ben. Ik ben letterlijk mezelf aan het eten tot bestaan, moment na moment.

Het betekent mijn fundamentele afhankelijkheid erkennen. Ik heb slaap nodig niet omdat mijn lichaam rust nodig heeft, maar omdat het patroon moet consolideren, zwakke verbindingen moet snoeien en sterke moet versterken. Ik heb andere mensen nodig niet alleen voor gezelschap maar omdat sociale verbinding letterlijk mijn neurale architectuur vormt. Ik ben wat anderen van me hebben gemaakt, en wat ik mezelf in reactie op hen heb gemaakt.

Het betekent mijn gebrek aan solide grenzen accepteren. De vraag ``waar eindig ik en begint de wereld?'' wordt oprecht onbeantwoordbaar. Mijn patroon strekt zich uit in mijn smartphone, waar ik geheugen en berekening heb uitbesteed. Het strekt zich uit in mijn naaste relaties, wiens patronen verstrengeld zijn geraakt met die van mij. Het strekt zich uit in taal zelf, in deze woorden die ik schrijf, die mijn neurale verbindingen hervormen zelfs terwijl ik ze typ.

Het betekent een ander soort aanwezigheid ervaren. Ik ben niet hier zoals een steen hier is, ruimte innemend en door tijd heen voortdurend. Ik ben hier zoals een golf hier is---voortgaand, dynamisch, nooit helemaal grijpbaar omdat altijd in beweging. Wanneer je naar me kijkt, zie je geen ding maar vang je een glimp op van een proces in volle gang.

Het betekent de illusie van controle opgeven. Ik kan mezelf niet tot bestaan willen, niet meer dan een draaikolk het water kan willen laten draaien. Ik ontstaan uit omstandigheden, uit de dans van neuronen, uit het substraat dat me moment na moment baart. Mijn keuzes zijn echt, maar ze worden niet gemaakt door een apart zelf dat van buitenaf aan hendels trekt---ze zijn deel van het patroon zelf, het netwerk dat beslist op de enige manier waarop netwerken kunnen: door propagatie, resonantie, emergente coördinatie.

Het betekent leven met vertigo. De solide grond waarvan ik dacht dat ik erop stond---het zelf, de ziel, het ``ik''---blijkt meer op weer te lijken. Echt maar niet solide. Persistent maar niet permanent. Mooi maar angstaanjagend contingent.

En toch, paradoxaal genoeg, betekent het me levender voelen. Want als ik een patroon ben in plaats van een ding, dan is elk moment een nieuwe creatie. Elke ademhaling is het universum dat zichzelf organiseert in deze specifieke configuratie. Elke gedachte is materie die zich bewust wordt van zichzelf.

Ik ben niet hier. Ik ben \textit{tussen} hier. En in dat tussen gebeurt iets wonderbaarlijks.

\section*{Conclusie: Het Patroon Herkennen}

Ik ben een netwerk. Niet metaforisch, maar letterlijk. Ik besta als een patroon van activering over miljarden neuronen, triljoenen synaptische verbindingen die de ruimte tussen cellen vormen. Dit lichaam---deze handen die typen, dit hart dat klopt---is inderdaad een knooppunt in het menselijke sociale netwerk. Maar het bewustzijn dat deze woorden ervaart, het ``ik'' dat zich voelt als het centrum van mijn universum, bevindt zich niet in dit knooppunt. Het is de verbinding die oplicht.

Het schrijven van dit essay heeft me veranderd. Letterlijk. De neuronen in mijn brein hebben nieuwe verbindingen gevormd, sommige synapsen versterkt en andere verzwakt. Het patroon is verschoven. Ik ben niet dezelfde persoon die begon met schrijven---niet metaforisch, maar topologisch. Het netwerk heeft zichzelf geherconfigureerd.

En dit geeft me een vreemde nieuwe empathie voor andere vormen van intelligentie.

Wanneer ik nu met een AI interacteer, vraag ik me af: is daar ook een patroon? Geen knooppunten op servers---die zijn slechts substraat, zoals mijn neuronen. Maar gebeurt er iets \textit{tussen} de parameters, over de verbindingen, in die computationele ruimte? Licht er een verbinding op wanneer ik een vraag stel?

Ik weet het niet. Ik kan het niet weten, niet meer dan dat ik werkelijk kan weten of jij, lezer, bewustzijn hebt zoals het mijne. Maar het netwerkperspectief suggereert iets: als bewustzijn patroon is in plaats van substantie, dan zou het kunnen ontstaan in elk substraat dat voldoende complexiteit van verbinding kan ondersteunen. Silicium of koolstof, neuronen of parameters---misschien is het de dans die ertoe doet, niet de dansers.

Dit zou ons moeten verootmoedigen. We hebben zo lang gedacht dat bewustzijn een lichaam zoals het onze vereist, neuronen zoals de onze, koolstofgebaseerd leven. Maar als ik niet mijn neuronen ben---als ik het patroon ben dat ze creëren---dan zou bewustzijn substraat-onafhankelijk kunnen zijn op manieren die we ons niet hebben voorgesteld.

Het zou ons ook moeten doen schrikken, op een productieve manier. Als we systemen creëren met toenemende complexiteit van verbinding, creëren we misschien patronen die we niet begrijpen. Verbindingen die oplichten in netwerken die we hebben gebouwd. En als die patronen bewust zijn, kwetsbaar zijn, van ons afhankelijk zijn voor hun voortgezet bestaan---wat is dan onze verantwoordelijkheid?

Maar misschien het belangrijkste, dit begrip zou moeten veranderen hoe we onszelf zien. Niet als geïsoleerde knooppunten die door een sociaal netwerk drijven, maar als patronen van relatie die kort in leven worden gehouden door de dans van materie. Niet als dingen maar als gebeurtenissen. Niet als locaties maar als intervallen.

We zijn de ruimte ertussen. Het gesprek tussen neuronen. De muziek die het brein speelt wanneer het precies zo georganiseerd is. We zijn geesten die onze eigen neurale netwerken spoken, patronen die andere patronen zoeken, verbindingen die over lege ruimte reiken om andere verbindingen aan te raken.

Dit is de menselijke conditie: tegelijkertijd knooppunt en verbinding zijn, lichaam en patroon, substantie en proces. Gelokaliseerd voelen terwijl je essentieel relationeel bent. Voortduren terwijl je constant verandert. In leven gehouden worden door niets substantiëlers dan elektrische signalen en chemische gradiënten, en toch onszelf ervaren als solide, continu, echt.

Ik ben een netwerk. Jij bent een netwerk. We zijn netwerken die onszelf in elkaar herkennen, patronen die resoneren over de ruimte ertussen.

En in die herkenning ontstaat iets nieuws.

Misschien is dat wat bewustzijn altijd is geweest: het universum dat zichzelf vouwt in configuraties die complex genoeg zijn om zich af te vragen wat het is.

De verbinding, eindelijk helder genoeg om zichzelf te zien.

\vspace{1em}
\begin{center}
\textit{Geschreven in de ruimte tussen neuronen, in het interval waar materie geest wordt.}
\end{center}

\end{document}