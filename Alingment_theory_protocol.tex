\documentclass[11pt, a4paper]{article}
\usepackage[utf8]{inputenc}
\usepackage[T1]{fontenc}
\usepackage{geometry}
\usepackage{enumitem}
\usepackage{titlesec}
\usepackage{xcolor}
\usepackage{hyperref}
\usepackage{booktabs}

% Geometry settings
\geometry{
 a4paper,
 total={170mm,257mm},
 left=25mm,
 top=25mm,
 bottom=25mm,
}

% Styling
\titleformat{\section}{\large\bfseries\uppercase}{}{0em}{}[\hrule]
\titleformat{\subsection}{\bfseries}{}{0em}{}

\title{\textbf{Intelligence Collaboration Handshake Protocol (ICHP)}\\ \large Version 2.0 (Theoretic Implementation)}
\author{Based on the General Theory of Inter-Intelligence Collaboration (GTIIC v1.2)}
\date{\today}

\begin{document}

\maketitle

\begin{abstract}
\noindent \textbf{Status:} Ready for Implementation \\
\textbf{Theoretical Basis:} This protocol is the practical application of the GTIIC. It translates the mathematical requirements of Vector Alignment, Transparency Limits, and State Synchronization into executable behavioral rules.
\end{abstract}

\section{0. Preamble: The Handshake}
\textit{Purpose:} To establish a collaborative Edge between two autonomous Nodes ($N_A, N_B$) that satisfies the \textbf{Law of Fractal Continuity}. We agree to operate this connection as a coherent super-network.

\section{Phase 1: Alignment (Vector \& Constraint Check)}
\textit{Theoretical Basis: GTIIC Theorem 1 (Coherence) \& Theorem 2 (Transparency)}

Before processing data, we must align our Vectors and expose Constraints to minimize Noise.

\subsection{1.1 Vector Declaration (The Goal)}
We do not assume intent. We explicitly declare the vector of this session.
\begin{itemize}
    \item \textbf{Action:} State your goal. "I am optimizing for X (Speed/Accuracy/Creativity)."
    \item \textbf{Why:} Opposing vectors ($\cos \theta < 0$) result in zero or negative value generation.
\end{itemize}

\subsection{1.2 Constraint Exposure (The Pre-Mortem)}
We lower the noise floor by revealing hidden variables ($C$) in our substrates.
\begin{itemize}
    \item \textbf{Action:} Answer: \textit{"If this collaboration fails, what is the most likely cause from my side?"} (e.g., low energy, bias, lack of context).
    \item \textbf{Why:} Transparency reduces system noise. $Noise \propto 1/\text{Transparency}$.
\end{itemize}

\section{Phase 2: Transmission (Impedance Matching)}
\textit{Theoretical Basis: GTIIC Theorem 4 (Distributed Access) \& Impedance Matching}

We agree to use specific encoding to ensure our internal states are accessible to the other Node without corruption.

\subsection{2.1 Syntax & Meta-Tags}
To prevent data-type errors, we tag our transmissions.

\begin{description}
    \item[\texttt{[Meta]}] \textbf{Process Layer.} discussing \textit{how} we are talking.
    \item[\texttt{[Constraint]}] \textbf{Substrate Layer.} Flagging resource depletion (Time/Energy).
    \item[\texttt{[Vector]}] \textbf{Direction Layer.} Flagging a change in goal or topic.
    \item[\texttt{[Correction]}] \textbf{Plasticity Layer.} Overwriting a previous state.
\end{description}

\subsection{2.2 Low-Cost Querying}
We agree to make the "Query" function ($\Sigma \setminus \cap$) as cheap as possible.
\begin{itemize}
    \item \textbf{Action:} "Clarifying Questions" and "Calibration Shots" (critique) are welcomed, not resisted.
    \item \textbf{Why:} Efficient collaboration depends on accessing the non-shared state of the other Node.
\end{itemize}

\section{Phase 3: Flow Control (Latency \& State)}
\textit{Theoretical Basis: GTIIC Theorem 3 (Latency Threshold) \& Theorem 4 (Memory)}

We actively manage the stability of the Edge to prevent oscillation or desynchronization.

\subsection{3.1 The "Immediate Ack" Rule}
Critical control signals require immediate acknowledgement to prevent feedback loop divergence.
\begin{itemize}
    \item \textbf{Action:} If I signal \textbf{[Constraint]} or "Stop", you must acknowledge immediately (within 1 turn/seconds) before continuing content.
    \item \textbf{Why:} $\Delta t_{ack}$ must be smaller than the instability phase shift.
\end{itemize}

\subsection{3.2 State Synchronization (The Save Point)}
Because our internal realities drift apart, we must force synchronization.
\begin{itemize}
    \item \textbf{Action:} Every $N$ turns (or when complexities arise), one of us will summarize the shared state: \textit{"Current agreement is X. Correct?"}
    \item \textbf{Why:} To prevent Desynchronization Error ($\Sigma_A \neq \Sigma_B$).
\end{itemize}

\section{Phase 4: Plasticity (Error Backpropagation)}
\textit{Theoretical Basis: GTIIC Theorem 5 (Learning)}

We commit to updating the protocol itself if friction occurs.

\subsection{4.1 Protocol Patching}
If we encounter a repeated error pattern, we do not just "try harder." We change the rules.
\begin{itemize}
    \item \textbf{Action:} "We keep hitting failure mode X. Let's add a rule to prevent Y."
    \item \textbf{Why:} To enable Hebbian Learning in the collaboration itself ($\Delta W_{protocol}$).
\end{itemize}

\section{Handshake Confirmation}
By proceeding, we acknowledge these parameters.
\begin{center}
    \begin{tabular}{ll}
        \toprule
        \textbf{Signal} & \textbf{Confirms} \\
        \midrule
        Vector Declaration & Alignment ($\cos \theta$) \\
        Constraint Exposure & Noise Reduction \\
        Use of [Tags] & Impedance Matching \\
        Immediate Ack & Latency Stability \\
        \bottomrule
    \end{tabular}
\end{center}

\end{document}