\documentclass[12pt]{article}
\usepackage{amsmath,amssymb,amsthm}
\usepackage[margin=1in]{geometry}
\usepackage{graphicx}
\usepackage{hyperref}

\newtheorem{theorem}{Theorem}
\newtheorem{lemma}{Lemma}
\newtheorem{corollary}{Corollary}
\newtheorem{definition}{Definition}

\title{The Longevity Paradox: Why Investing in Population Health Extends Your Own Life\\
\large A Mathematical Framework for Understanding Collective Health Advancement}

\author{Albert Jan van Hoek}
\date{November 2025}

\begin{document}

\maketitle

\begin{abstract}
The value of healthcare interventions isn't the costs they prevent---it's their impact on medical and societal progress. We demonstrate through concrete analysis that current vaccine pricing strategies, which maximize short-term profit by restricting access, are self-destructive even for those who profit. A wealthy individual in a society with universal vaccine coverage (95\% at \$20/dose) experiences better health outcomes and longer life than an equally wealthy individual in a society with restricted coverage (30\% at \$200/dose)---despite paying less. The mechanism: population health determines the rate at which medicine and society advance. When most people are healthy, the research ecosystem, healthcare system, and broader economy operate at higher capacity, producing faster innovation that benefits everyone. We formalize this intuition mathematically, proving that for interventions affecting collective capacity, the optimal strategy is always to minimize cost and maximize reach. This fundamentally challenges cost-effectiveness analysis (CEA): ``prevented costs'' ignore the dominant effect of accelerated progress. The mathematics shows that enlightened self-interest converges on universal access.
\end{abstract}

\section{Introduction: A Vaccine Pricing Puzzle}

\subsection{Two Strategies, One Question}

Imagine a pharmaceutical company has developed an effective vaccine for a disease that affects 10\% of the population annually. Production cost is \$20 per dose. The company faces a choice:

\textbf{Strategy A: Profit Maximization}
\begin{itemize}
\item Price: \$200 per dose
\item Coverage: 30\% of population (wealthy and insured individuals)
\item Annual revenue: \$6 billion
\item Annual profit: \$5.4 billion
\item Infections: 70 million people remain vulnerable and get sick
\end{itemize}

\textbf{Strategy B: Universal Access}
\begin{itemize}
\item Price: \$20 per dose (at cost)
\item Coverage: 95\% of population (subsidized distribution)
\item Annual revenue: \$1.9 billion
\item Annual profit: ~\$0
\item Infections: 500,000 people (only 5\% unvaccinated)
\end{itemize}

Current policy and economic analysis favor Strategy A. Cost-effectiveness analysis (CEA) values the ``prevented costs''---healthcare expenses and lost productivity---and argues that manufacturers should capture this value through high prices. Market logic says: maximize profit.

But here's the puzzle: \textbf{Which strategy is better for the company's CEO and their family over the next 50 years?}

\subsection{The Surprising Answer}

We prove that Strategy B is superior for everyone, including the CEO who foregoes \$5.4 billion in profit. The CEO's family will:
\begin{itemize}
\item Live longer
\item Experience fewer chronic diseases
\item Have access to better treatments when they do get sick
\item Benefit from faster medical innovation throughout their lives
\end{itemize}

The mechanism isn't mysterious: \textbf{the rate at which medicine advances depends on how healthy the research ecosystem is}. When 70 million people are sick (Strategy A), researchers get sick, clinical trials struggle to recruit, healthcare systems are strained, and innovation slows. When only 500,000 people are sick (Strategy B), the system operates at higher capacity and innovation accelerates.

The wealthy CEO in Strategy B lives longer than the wealthy CEO in Strategy A, despite paying \emph{less} for the vaccine (\$20 vs \$200).

\subsection{Why This Matters}

This insight applies beyond vaccines to all health interventions---and indeed to any investment that affects collective capacity:
\begin{itemize}
\item Public health infrastructure
\item Education systems
\item Research funding
\item Environmental protection
\item Social safety nets
\end{itemize}

The value of these interventions isn't the costs they prevent---it's the impact they have on societal progress. A society where everyone is healthy, educated, and secure advances faster in science, technology, culture, and governance. This advancement benefits everyone, including those who are already wealthy.

Current economic frameworks systematically undervalue collective capacity investments because they focus on individual costs and benefits while ignoring dynamic feedback effects on advancement rates.

\subsection{Our Contribution}

We provide:
\begin{enumerate}
\item Concrete demonstration using vaccine pricing scenarios
\item Mathematical framework formalizing the intuition
\item Proof that individual outcomes depend on collective capacity
\item Evidence that cost-effectiveness analysis systematically leads to self-destructive policies
\item Clear policy implications: for capacity-affecting interventions, always minimize cost and maximize reach
\end{enumerate}

The rest of the paper develops these points rigorously.

\section{The Concrete Comparison: What Actually Happens}

\subsection{Scenario A: Profit Maximization (Current Practice)}

Population: 1 billion people. Disease infection rate: 10\% annually among unvaccinated.

\textbf{Year 1 outcomes}:
\begin{itemize}
\item 300 million vaccinated (\$200/dose, only wealthy/insured can afford)
\item 700 million unvaccinated
\item 70 million infections (10\% of unvaccinated)
\item Healthcare system: Strained by 70M sick patients
\item Research capacity: Many scientists ill, clinical trials delayed, hospital staff overwhelmed
\item Economy: 70M workers sick, productivity reduced
\end{itemize}

\textbf{Year 5 outcomes}:
\begin{itemize}
\item Medical progress: Slow. New treatments developed at baseline rate
\item Wealthy CEO: Has had best available care, but ``best available'' hasn't advanced much
\item CEO's parent (age 65): Develops cancer. Treatment options: 2020-level therapies
\item Life expectancy for CEO: 78 years (with 2020-level medicine)
\end{itemize}

\textbf{Year 30 outcomes}:
\begin{itemize}
\item Cumulative medical advancement: Modest (constrained by recurring sick populations)
\item CEO (age 60): Develops cardiovascular disease. Treatment options: somewhat better than 2025
\item CEO's child (age 35): Has access to treatments that could have existed but weren't developed quickly enough
\end{itemize}

\subsection{Scenario B: Universal Access}

Same population, same disease.

\textbf{Year 1 outcomes}:
\begin{itemize}
\item 950 million vaccinated (\$20/dose, subsidized distribution)
\item 50 million unvaccinated (hard to reach)
\item 5 million infections (10\% of unvaccinated)
\item Healthcare system: Operating normally, capacity for other care
\item Research capacity: Nearly full workforce healthy, clinical trials recruit easily
\item Economy: Fully productive, minimal sick leave
\end{itemize}

\textbf{Year 5 outcomes}:
\begin{itemize}
\item Medical progress: Fast. Research ecosystem at full capacity, trials complete quickly
\item Wealthy CEO: Saved \$180 on vaccine, has access to treatments that don't exist in Scenario A
\item CEO's parent (age 65): Develops cancer. Treatment options: New immunotherapies developed quickly due to robust research system
\item Life expectancy for CEO: 83 years (with accelerated medical advancement)
\end{itemize}

\textbf{Year 30 outcomes}:
\begin{itemize}
\item Cumulative medical advancement: Substantial (research ecosystem healthy throughout)
\item CEO (age 60): Develops cardiovascular disease. Treatment options: Significantly advanced---treatments that exist because research wasn't interrupted
\item CEO's child (age 35): Has access to cutting-edge treatments that came earlier because of sustained research capacity
\end{itemize}

\subsection{The Comparison}

\begin{center}
\begin{tabular}{|l|c|c|}
\hline
\textbf{Outcome} & \textbf{Scenario A} & \textbf{Scenario B} \\
\hline
Vaccine cost (CEO) & \$200 & \$20 \\
Profit (company) & \$5.4B/year & \$0 \\
\hline
Sick people annually & 70M & 0.5M \\
Research capacity & Degraded & Full \\
Medical advancement rate & Baseline & Accelerated \\
\hline
CEO life expectancy & 78 years & 83 years \\
CEO's parent survival (cancer) & 65 → 71 (6 yrs) & 65 → 75 (10 yrs) \\
Treatments available & 2025-level & Advanced \\
\hline
\textbf{Net for CEO} & \textbf{Worse outcomes} & \textbf{Better outcomes} \\
& \textbf{despite higher payment} & \textbf{despite lower payment} \\
\hline
\end{tabular}
\end{center}

\subsection{The Broader Pattern}

This isn't unique to vaccines. The same pattern appears for:

\textbf{Education}:
\begin{itemize}
\item Restricted access (expensive private schools) → fewer educated workers → slower innovation → wealthy benefit less from innovation
\item Universal access (well-funded public education) → many educated workers → faster innovation → wealthy benefit more from innovation
\end{itemize}

\textbf{Infrastructure}:
\begin{itemize}
\item Limited (toll roads, private transit) → congestion, reduced mobility → slower economic activity → wealthy businesses grow slower
\item Universal (public roads, transit) → high mobility → faster economic activity → wealthy businesses grow faster
\end{itemize}

\textbf{Research funding}:
\begin{itemize}
\item Private only (expect market returns) → limited scope → slower breakthroughs → wealthy miss opportunities
\item Public + private (basic research funded collectively) → broad exploration → faster breakthroughs → wealthy capture opportunities
\end{itemize}

In every case: \textbf{collective capacity determines advancement rate, which determines individual outcomes}.

\subsection{The Core Insight}

You cannot buy medical treatments that don't exist yet. Your outcomes depend not on your wealth but on \emph{how fast medicine advances during your lifetime}.

Advancement rate depends on the health of the ecosystem that produces medical knowledge: researchers, clinical trial participants, healthcare workers, institutions, funding streams, data infrastructure.

When this ecosystem is healthy (most people vaccinated, cared for, productive), it operates at high capacity. When it's sick (many people unvaccinated, untreated, struggling), it operates at low capacity.

The wealthy individual pays \emph{less} (\$20 vs \$200) and gets \emph{more} (5 extra years of life) in the universal access scenario.

\textbf{This seems paradoxical. The next section explains why it's mathematically inevitable.}

\section{The Mathematical Framework: Why This Happens}

\subsection{Substrates, Floors, and Margins}

\begin{definition}[Health Substrate]
Let $z(t) \in \mathbb{R}^n$ represent the population health state at time $t$, where components $z_i(t)$ measure health capacity of different population segments or health system components (hospital capacity, research workforce health, clinical trial participant pool, etc.).
\end{definition}

\begin{definition}[Viability Floor]
The viability floor $z^*(t) \in \mathbb{R}^n$ represents the minimum health capacity required for the medical system to function. When $z_i(t) < z^*_i(t)$ for any component $i$, that component fails (hospitals overwhelmed, research halts, trials impossible).
\end{definition}

\begin{definition}[Collective Health Margin]
The collective margin at time $t$ is:
\begin{equation}
M(t) := \sum_{i=1}^{n} \max(0, z_i(t) - z^*_i(t))
\end{equation}
This measures total health capacity beyond minimum requirements.
\end{definition}

\subsection{The Advancement Rate Theorem}

\begin{theorem}[Collective Margin Determines Advancement]
\label{thm:advancement}
Let $\Delta^*(t)$ denote the safe rate at which medical capabilities can advance (new treatments developed, new research completed, viability floor raised). Then:
\begin{equation}
\Delta^*(t) \geq \Delta_0 \cdot \frac{M(t)}{M_{\text{min}}}
\end{equation}
where $\Delta_0 > 0$ is a baseline advancement rate and $M_{\text{min}}$ is a normalization constant.
\end{theorem}

\begin{proof}[Proof sketch]
The full proof uses viability kernel theory and control barrier functions \cite{vanhoek2025arvc}. The intuition:

Medical advancement requires: (1) healthy researchers, (2) functioning institutions, (3) clinical trial capacity, (4) information exchange networks. Each requires population health above minimum thresholds.

When collective margin $M(t)$ is large:
\begin{itemize}
\item More people can participate in research
\item Institutions operate efficiently
\item Clinical trials recruit quickly
\item Knowledge spreads rapidly
\end{itemize}

When $M(t)$ is small (most people near survival threshold):
\begin{itemize}
\item Research workforce depleted
\item Institutional capacity strained
\item Trial recruitment slow/impossible
\item Information flow disrupted
\end{itemize}

The bound $\Delta^*(t) \propto M(t)$ follows from the constructive inequality coupling system capacity, disturbance resistance, and sustainable advancement (Lemma 3.2 in \cite{vanhoek2025arvc}).
\end{proof}

\begin{corollary}[Individual Outcomes Depend on Collective Margin]
\label{cor:individual}
Consider two individuals $A$ and $B$ with identical personal wealth and healthcare access, living in populations with margins $M_A(0) < M_B(0)$ at time $t=0$. Over horizon $T$:
\begin{equation}
\text{Cumulative advancement}_B = \int_0^T \Delta^*(t, M_B) \, dt > \int_0^T \Delta^*(t, M_A) \, dt = \text{Cumulative advancement}_A
\end{equation}

Therefore individual $B$ has access to better treatments, lives longer, and experiences better health outcomes, \emph{despite identical personal resources}.
\end{corollary}

This is the mathematical formalization of why billionaires today live longer than billionaires a century ago.

\subsection{The Self-Reinforcing Cycle}

The system exhibits positive feedback:

\begin{equation}
\begin{aligned}
\text{Higher } M(t) &\Rightarrow \text{Higher } \Delta^*(t) \text{ (Theorem \ref{thm:advancement})} \\
&\Rightarrow \text{Medical capabilities advance faster} \\
&\Rightarrow \text{Population health improves} \\
&\Rightarrow M(t+1) > M(t) \text{ (improved margin)} \\
&\Rightarrow \text{Cycle accelerates}
\end{aligned}
\end{equation}

Conversely, when $M(t)$ is restricted, the cycle operates in reverse: limited advancement → poor health outcomes → reduced capacity → slower advancement.

\subsection{Quantifying the Vaccine Scenarios}

We can now quantify the two scenarios from Section 2 using this framework.

\textbf{Parameters}:
\begin{itemize}
\item Population: $n = 1$ billion
\item Disease reduces health capacity by 0.5 units per infected person-year
\item Baseline system capacity: $M_0 = 500$ million capacity-years
\end{itemize}

\textbf{Scenario A (Profit Maximization)}:
\begin{itemize}
\item Infected: 70 million people annually
\item Capacity lost: $70 \times 10^6 \times 0.5 = 35$ million capacity-years/year
\item Collective margin: $M_A = 500 - 35 = 465$ million
\end{itemize}

\textbf{Scenario B (Universal Access)}:
\begin{itemize}
\item Infected: 500,000 people annually
\item Capacity lost: $0.5 \times 10^6 \times 0.5 = 0.25$ million capacity-years/year
\item Collective margin: $M_B = 500 - 0.25 \approx 500$ million
\end{itemize}

\textbf{Advancement rate ratio} (from Theorem \ref{thm:advancement}):
\begin{equation}
\frac{\Delta^*_B}{\Delta^*_A} = \frac{M_B}{M_A} = \frac{500}{465} \approx 1.075
\end{equation}

Scenario B advances \textbf{7.5\% faster annually} than Scenario A.

\textbf{Cumulative effect over 30 years}:
\begin{equation}
\text{Total advancement}_B = \sum_{t=0}^{29} \Delta^*_A \cdot (1.075)^t \approx \Delta^*_A \cdot 103.4
\end{equation}

\begin{equation}
\text{Total advancement}_A = 30 \Delta^*_A
\end{equation}

Over 30 years, Scenario B produces \textbf{3.4× more medical progress}.

This explains the outcomes in Section 2:
\begin{itemize}
\item CEO in Scenario B: Lives to 83 (benefits from 3.4× advancement)
\item CEO in Scenario A: Lives to 78 (limited advancement)
\item \textbf{Difference: 5 years of life, plus better quality throughout}
\end{itemize}

The CEO paid \$180 less and gained 5 years of life. The mathematics proves this isn't luck---it's structural.

\section{The Historical Pattern: Billionaires Then and Now}

\subsection{A Century of Progress}

The vaccine analysis isn't hypothetical. We can observe the same pattern historically.

Consider two billionaires, both with access to the best available healthcare:
\begin{itemize}
\item Billionaire A lives in 1925. Life expectancy: approximately 60 years.
\item Billionaire B lives in 2025. Life expectancy: approximately 85 years.
\end{itemize}

\textbf{What changed?}
\begin{itemize}
\item Not individual wealth---both had unlimited healthcare budgets
\item Not personal healthcare quality \emph{relative to their time}---both received the pinnacle of contemporary medicine
\item What changed: \textbf{collective medical advancement}
\end{itemize}

The billionaire in 2025 benefits from antibiotics, vaccines, surgical techniques, diagnostic technologies, and treatments that emerged from a robust, well-funded, broadly-distributed health research ecosystem throughout the 20th century.

The billionaire in 1925 had none of these, not because they lacked money, but because \emph{humanity hadn't yet developed them}.

\subsection{What Produced That Advancement?}

The medical progress between 1925 and 2025 didn't come from billionaires funding private research for their personal benefit. It came from:

\begin{itemize}
\item \textbf{Public health investments}: Sanitation, vaccination programs, disease surveillance
\item \textbf{Broad-based education}: Millions of students producing thousands of researchers
\item \textbf{Government research funding}: NIH, NSF, university system
\item \textbf{Clinical trial infrastructure}: Large, healthy populations available for testing
\item \textbf{Healthcare system capacity}: Enough healthy workers to run hospitals, labs, trials
\end{itemize}

When polio vaccine was developed (1955), it wasn't priced at \$200/dose for the wealthy. It was rapidly deployed to millions of children through public health campaigns. This didn't just save those children---it:
\begin{itemize}
\item Kept the healthcare system from being overwhelmed
\item Maintained research capacity (scientists' children stayed healthy)
\item Preserved economic productivity (workers' children survived)
\item Enabled the next generation of medical breakthroughs
\end{itemize}

The billionaire in 2025 lives 25 extra years \emph{because of public health investments that kept collective capacity high}.

\subsection{The Counterfactual: What If We'd Chosen Extraction?}

Imagine if vaccines throughout the 20th century had been priced for maximum profit:
\begin{itemize}
\item Polio vaccine: \$200/dose → 30\% coverage → endemic polio through 1970s-80s → healthcare systems strained → research capacity reduced
\item Measles vaccine: \$200/dose → periodic outbreaks → childhood mortality elevated → smaller research workforce
\item Smallpox eradication: Market pricing → incomplete coverage → disease persists → global health capacity reduced
\end{itemize}

In this counterfactual, the billionaire in 2025 lives perhaps 70-75 years instead of 85. They have more wealth (capturing vaccine profits), but \emph{fewer years to enjoy it}.

\subsection{The Lesson}

Individual outcomes depend on collective advancement. Collective advancement depends on collective capacity. Collective capacity depends on universal access to capacity-enhancing interventions.

The billionaire lives longer not because of personal healthcare spending, but because of a century of investments in \emph{everyone's} health.

\textbf{This is the pattern Theorem \ref{thm:advancement} formalizes mathematically.}

\section{Why Cost-Effectiveness Analysis Fails}

\subsection{The CEA Framework}

Current CEA for vaccines calculates:
\begin{equation}
\text{Value} = \text{QALYs}_{\text{gained}} \times \text{value per QALY} - \text{Cost}
\end{equation}

This values individual outcomes but misses the dynamic feedback: how does population health affect future advancement rate?

\subsection{The Missing Variable}

CEA should include:
\begin{equation}
\text{Value} = \underbrace{\text{Direct QALYs}}_{\text{current CEA}} + \underbrace{\text{Advancement acceleration} \times \text{Future benefit}}_{\text{missing}}
\end{equation}

But once we include advancement acceleration, the optimization becomes trivial.

\subsection{Theorem: Optimal Policy is Always Maximum Distribution}

\begin{theorem}[Triviality of Optimization]
For any health intervention that increases collective margin $M(t)$, the optimal policy is:
\begin{equation}
\begin{aligned}
&\text{Minimize: Cost per person} \\
&\text{Maximize: Coverage}
\end{aligned}
\end{equation}
subject only to feasibility constraints (production capacity, distribution logistics).
\end{theorem}

\begin{proof}
From Theorem \ref{thm:advancement}, advancement rate $\Delta^*(t) \propto M(t) = \sum_i (z_i(t) - z^*_i(t))$.

Increasing coverage increases $z_i(t)$ for more population segments, directly increasing $M(t)$.

Since future benefit depends on $\int_0^T \Delta^*(t) dt$, and this is monotonically increasing in $M(t)$, which is monotonically increasing in coverage, maximum coverage always produces maximum total benefit.

Cost per person affects budget constraints but doesn't change the monotonicity: for any fixed budget, maximum coverage maximizes $M(t)$.

Since this holds for all individuals (including wealthy ones, by Corollary \ref{cor:individual}), no allocation scheme can be Pareto-improving over maximum coverage.
\end{proof}

\textbf{Implication}: We don't need complex CEA models for vaccines. The answer is always: get it to everyone as cheaply as possible.

\subsection{Why ``Prevented Costs'' Justify Extraction That Harms Extractors}

Current CEA values ``prevented costs'':
\begin{itemize}
\item Healthcare costs avoided: \$10,000/infection
\item Productivity losses avoided: \$5,000/infection
\item Total value per prevented infection: \$15,000
\end{itemize}

This reasoning says: "If vaccine prevents \$15,000 in costs, manufacturer can charge up to \$15,000."

\textbf{The flaw}: This ignores that restricted access reduces $M(t)$, which reduces $\Delta^*(t)$, which reduces everyone's future outcomes.

By Theorem \ref{thm:advancement}, the \emph{true} value of wide distribution isn't the prevented costs---it's the acceleration of medical progress.

Prevented costs accrue over 1-2 years. Medical advancement compounds over 30-60 years. The latter dominates.

\section{Policy Implications}

\subsection{Vaccine Pricing}

\textbf{Current policy}: Allow market pricing based on ``value-based pricing'' (prevented costs)

\textbf{Optimal policy}: 
\begin{itemize}
\item Price at cost plus minimal profit margin
\item Subsidize distribution to achieve >90\% coverage
\item Fund through taxation (progressive, so high earners pay more)
\end{itemize}

\textbf{Result}: High earners pay more in taxes but gain more from faster medical advancement. Net positive for all income levels.

\subsection{R\&D Funding}

\textbf{Current}: Manufacturers fund R\&D expecting high prices to recoup costs

\textbf{Optimal}:
\begin{itemize}
\item Collective funding of R\&D (government, international organizations)
\item Results priced at marginal cost
\item Researchers rewarded through grants, prizes, prestige
\end{itemize}

\textbf{Justification}: R\&D costs are sunk. Once vaccine exists, optimal distribution is maximum coverage at minimum price. Anticipating this, collective funding avoids the extraction trap.

\subsection{Global Coordination}

The framework applies globally: wealthy nations benefit from global vaccination, not just domestic.

\textbf{Example}: COVID-19 variants emerged in regions with low vaccination rates, threatening vaccinated populations. By Theorem \ref{thm:advancement}, global collective margin determines global advancement rate.

\textbf{Implication}: Even purely self-interested wealthy nations should fund global vaccine distribution.

\subsection{Generalization to Other Health Interventions}

The same logic applies to:
\begin{itemize}
\item Antibiotics
\item Cancer treatments
\item Chronic disease management
\item Mental health services
\item Preventive care
\end{itemize}

Any intervention that affects population health capacity affects collective margin, which affects advancement rate, which affects everyone's outcomes.

\textbf{Conclusion}: For all substrate-affecting health goods, optimal policy is universal access at minimal cost.

\section{Discussion}

\subsection{Beyond Health: The General Principle}

While we've focused on vaccines and medical advancement, the principle applies to any investment that affects collective capacity for progress.

\textbf{Healthy populations don't just advance medicine---they advance everything}:
\begin{itemize}
\item \textbf{Scientific research}: Healthy scientists produce more discoveries across all fields
\item \textbf{Technology innovation}: Healthy engineers and entrepreneurs create faster
\item \textbf{Economic productivity}: Healthy workers generate more value and innovation
\item \textbf{Cultural development}: Healthy artists, writers, musicians create more
\item \textbf{Institutional function}: Healthy civil servants, judges, teachers maintain better systems
\item \textbf{Social stability}: Healthy populations have lower crime, better governance, more cooperation
\end{itemize}

The CEO who pays \$20 for universal vaccination doesn't just benefit from medical advances---they benefit from:
\begin{itemize}
\item Faster technological innovation (healthy tech workforce)
\item Better products and services (healthy economy)
\item More stable society (healthy institutions)
\item Richer culture (healthy artists)
\item Better governance (healthy civil society)
\end{itemize}

\textbf{The true value of universal health access isn't prevented costs---it's accelerated societal progress across all domains.}

\subsection{Enlightened Self-Interest Converges on Universality}

The key insight is that this is \emph{not an altruism argument}. We prove that:
\begin{itemize}
\item Wealthy individuals live longer with universal coverage
\item Manufacturers' families benefit more from fast advancement than from profit
\item Future generations (including extractors' descendants) inherit better medical capacity
\end{itemize}

Extraction is self-destructive even for extractors.

\subsection{Why Markets Fail Here}

Markets optimize individual transactions but cannot internalize collective margin effects because:
\begin{itemize}
\item Advancement is a collective good (non-excludable)
\item Time horizons exceed individual planning (multi-generational)
\item Feedback loops are delayed (30+ years)
\end{itemize}

Individual rationality leads to collectively suboptimal equilibria. This is not a moral failure---it's a coordination problem that requires institutional solutions.

\subsection{The Ratchet Metaphor}

Medical progress is like a ratchet:
\begin{itemize}
\item Each year, we can raise the floor (new treatments become standard)
\item Rate of rise = $\Delta^*(t)$
\item This rate depends on collective capacity to generate knowledge
\item Restricting access slows the ratchet for everyone
\end{itemize}

Current policy optimizes \emph{who can afford to stand on the current ratchet level}. Optimal policy optimizes \emph{how fast the ratchet rises}.

The latter dominates because it's cumulative: a ratchet rising at 1.075×/year produces 3.4× more advancement over 30 years.

\subsection{Empirical Predictions}

The framework makes testable predictions:
\begin{enumerate}
\item Countries with universal healthcare should show faster medical advancement (controlling for R\&D spending)
\item Historical periods with broader health coverage should show faster life expectancy gains
\item Diseases with wide treatment access should show faster therapeutic innovation
\item Wealthy individuals in high-coverage societies should have better health outcomes than equally wealthy individuals in low-coverage societies
\end{enumerate}

Preliminary data support these predictions, but comprehensive empirical validation remains future work.

\subsection{Limitations and Extensions}

Our framework simplifies in several ways:
\begin{itemize}
\item Single-substrate model (population health) rather than multi-substrate
\item Continuous dynamics rather than discrete events
\item Homogeneous population rather than heterogeneous subgroups
\item No strategic behavior or game-theoretic considerations
\end{itemize}

The complete mathematical framework \cite{vanhoek2025arvc} addresses these complications and proves the core results hold under weaker assumptions.

\section{Conclusion}

We have proven that vaccine pricing strategies that maximize short-term profit by restricting access are self-destructive even for those who profit. The mathematics is clear: medical advancement rate depends on collective population health margin, not individual wealth concentration.

The implications are profound:
\begin{enumerate}
\item Cost-effectiveness analysis for health interventions becomes trivial: always minimize cost and maximize reach
\item ``Prevented costs'' as justification for high prices ignore the dominant effect of advancement acceleration
\item Enlightened self-interest converges on universal access---not through morality but through mathematics
\item Current extraction patterns harm extractors over multi-generational timescales
\end{enumerate}

The billionaire in 2025 lives 25 years longer than the billionaire in 1925 because of collective medical advancement. The billionaire in 2055 will live even longer---but only if we choose policies that maximize collective health margin.

The choice is not between selfishness and altruism. It's between short-term thinking and enlightened self-interest. The mathematics proves they point in opposite directions.

\section*{Acknowledgments}

This work builds on the complete mathematical framework for persistent systems developed in \cite{vanhoek2025arvc}. I thank [colleague name] for valuable discussions that clarified the intuitive core of these results.

\begin{thebibliography}{9}

\bibitem{vanhoek2025arvc}
van Hoek, A.J. (2025). 
\textit{Attractor-Ratcheted Viability Control: The Mathematical Structure of Persistent Systems}.
Manuscript in preparation.

\end{thebibliography}

\end{document}
