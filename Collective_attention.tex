\documentclass[11pt]{article}

\usepackage[a4paper,margin=2.5cm]{geometry}
\usepackage[T1]{fontenc}
\usepackage[utf8]{inputenc}
\usepackage{lmodern}
\usepackage{microtype}
\usepackage{hyperref}

\title{The Trajectory Commons (Idea)}
\author{}
\date{}

\begin{document}
\maketitle

\section*{A global planning commons / trajectory visualizer for humanity}

We are \textasciitilde8 billion people moving forward in time.

Whether we like it or not, our collective future is shaped by:
\begin{itemize}
    \item what we notice,
    \item what we worry about,
    \item what we hope for,
    \item what we prioritize,
    \item and what we actually do.
\end{itemize}

In other words: \textbf{attention shapes trajectory}.

But today, this process is mostly implicit and chaotic.
We have no shared interface where humanity can:
\begin{itemize}
    \item register what we are seeing,
    \item consolidate what we are worried about,
    \item compare what we want,
    \item explore solution pathways,
    \item and coordinate commitments.
\end{itemize}

\section*{Core idea}
Build a \textbf{Global Planning Commons / Trajectory Visualizer}:
a shared, open, machine-readable space where people can express future-related thoughts in natural language, and AI helps convert them into a navigable ``map of futures''.

This system is not ``one plan for the world''.

It is a \textit{map of possible trajectories} --- and a way to steer.

\section*{Why AI changes what is possible}
Before LLMs, we could not realistically consolidate billions of messy human statements into coherent structures.

Now we can.

LLMs can act as ``compilers'' for human intention:
\begin{itemize}
    \item extract observations, hopes, fears, and goals,
    \item cluster them into coherent themes,
    \item surface contradictions and dependencies,
    \item generate multiple candidate solution pathways,
    \item help transform attention into actionable commitments.
\end{itemize}

\section*{Separation of layers (epistemic hygiene)}
A crucial design principle is to keep different kinds of statements distinct:

\begin{enumerate}
    \item \textbf{Observations} (signals about the world)
    \item \textbf{Problem framings} (what is at stake)
    \item \textbf{Hypotheses} (causal beliefs about what might work)
    \item \textbf{Interventions} (candidate solutions)
    \item \textbf{Plans + commitments} (execution)
\end{enumerate}

This prevents ``solutionism'' and makes it possible to accumulate collective wisdom without prematurely locking into one narrative.

\section*{What the interface could feel like}
A user enters something like:
\begin{itemize}
    \item ``I'm worried about biodiversity collapse.''
    \item ``I want open-source health infrastructure to survive beyond a single maintainer.''
    \item ``I want resilience, diversity, and long-term continuity.''
\end{itemize}

The system responds with:
\begin{itemize}
    \item ``You are part of these shared concerns.''
    \item ``Here are related observations and alternative framings.''
    \item ``Here are multiple intervention families.''
    \item ``Here are existing initiatives and missing dependencies.''
    \item ``Here is where your commitment could matter.''
\end{itemize}

In other words: \textit{it turns isolated intention into collective motion.}

\section*{A deliberately outlandish comparison (that actually fits)}
This interface is somewhat like \textbf{praying to God}:
\begin{itemize}
    \item expressing fear and hope about the future,
    \item speaking into something larger than yourself,
    \item asking for a better trajectory.
\end{itemize}

Except here, the ``recipient'' is not supernatural.

It is a \textbf{shared coordination substrate}:
a system that can listen, consolidate, and route human intention into collective sensemaking and action.

\section*{What this could unlock}
If such a commons existed, people could start sharing:
\begin{itemize}
    \item what they're worried about,
    \item what they want to protect,
    \item what they want to build,
    \item what they can contribute.
\end{itemize}

Over time, humanity gains a new layer of capacity:
a way to make the future \textbf{legible}, \textbf{negotiable}, and \textbf{steerable}.

Not perfectly.\\
Not centrally.\\
But collectively --- with transparency, pluralism, and traceable assumptions.

\end{document}
