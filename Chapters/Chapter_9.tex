\chapter{Universal Evolution: Minerals as Evolving Networks} % Adjusted Title
\label{ch:UniversalEvolution}

Having reinterpreted biological fitness through the \emph{Evolution by Emergence} paradigm, we now explicitly test Principle 1 (Universality of Emergence) by examining evolution in a non-living system. This chapter uses the mineral kingdom as a case study, demonstrating how the paradigm's core concepts—dynamic networks (Principle 2), feedback loops (Principle 3), interdependence (Principle 4), abstract selection (Principle 5), and constrained agency (Principle 6)—apply beyond biology, driving complexity in the abiotic world. % Revised Introduction

\section*{Common Threads in Evolving Systems}

As proposed by the paradigm and supporting research \cite{WongEtAl2023}, many evolving systems, whether living or not, share common characteristics. They arise from components with vast combinatorial possibilities (Principle 8). They are subject to processes generating diverse configurations within a dynamic network (Principle 2). Crucially, selection mechanisms favor configurations exhibiting functions like stability or persistence (Principle 5, abstractly applied). Key traits include:
\begin{itemize}
    \item \textbf{Combinatorial Richness:} Systems are composed of diverse components (e.g., elements) that can combine in vast numbers of ways.
    \item \textbf{Configuration Generation:} Processes (e.g., geological changes) exist that explore and produce numerous different configurations (e.g., potential mineral formulas).
    \item \textbf{Selection for Function:} Mechanisms preferentially select configurations based on advantageous functions (e.g., thermodynamic stability leading to persistence).
\end{itemize}
This interplay between variation generation and selection, driven by feedback within the network (Principle 3), can drive systems toward increased complexity or patterning over time, irrespective of whether they are alive. % Added links to paradigm

\section*{Case Study: The Evolving Mineral Kingdom}

Earth's mineral kingdom offers a compelling, quantifiable case study supporting the paradigm's claim of universality (Principle 1) \cite{HazenWong2024}. Over billions of years, the diversity and complexity of minerals have systematically increased. This wasn't random; it was driven by Earth's changing physical and chemical conditions—an evolving geochemical network (Principle 2)—creating new environments and possibilities via feedback (Principle 3). The minerals that formed and persisted were effectively "selected" for their stability (an abstract form of competition/collaboration, Principle 5) within these conditions. Researchers quantify this using metrics like Functional Information (FI), tracking the rarity of stable configurations within the expanding possibility space. Studies show mineral FI monotonically increased throughout Earth's history \cite{HazenWong2024}, demonstrating directional evolution. Interestingly, this abiotic evolution appears "bounded"—approaching a limit defined by geochemical constraints (Principle 6: Constrained Agency)—potentially contrasting with biological evolution's open-ended nature \cite{HazenWong2024, WongEtAl2023}. This highlights how the paradigm accommodates different evolutionary trajectories (Principle 7, Principle 10). % Added links to paradigm

\section*{Universal Principles and the Emergence Paradigm}

This example of mineral evolution strongly supports the principle of \emph{Universality} (Principle 1) within the "Evolution by Emergence" paradigm. It demonstrates that the core mechanisms—emergence from interactions within dynamic networks (Principle 2), feedback loops driving change (Principle 3), interdependence within the geochemical context (Principle 4), abstract selection pressures (Principle 5), and constrained agency limiting possibilities (Principle 6)—are not confined to life. These principles appear fundamental to how complexity arises and transforms across diverse natural systems, requiring holistic, non-reductionist analysis (Principle 9) informed by complexity science (Principle 8). % Added links to paradigm

\section*{Conclusion}

Recognizing evolution in non-living systems like minerals, as framed by the Evolution by Emergence paradigm, significantly broadens our understanding of complexity and change in the universe (Principle 1, Principle 10). It reinforces the central theme of this book: that underlying principles of network dynamics and emergence shape the complex, interconnected systems we observe, from the geological to the biological and beyond. This universal perspective, demonstrated here with minerals, helps us appreciate the continuity of creative, evolutionary processes across different substrates and domains, governed by the paradigm's core principles. % Revised Conclusion
\cleardoublepage
