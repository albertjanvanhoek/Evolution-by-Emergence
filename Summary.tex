\documentclass[12pt,a4paper]{article}
\usepackage[utf8]{inputenc}
\usepackage[T1]{fontenc}
\usepackage[margin=1in]{geometry}
\usepackage{setspace}
\usepackage{titlesec}
\usepackage{hyperref}
\usepackage{amsmath}
\usepackage{amssymb}

\setstretch{1.15}

% Section formatting
\titleformat{\section}
  {\normalfont\Large\bfseries}{\thesection}{1em}{}
\titleformat{\subsection}
  {\normalfont\large\bfseries}{\thesubsection}{1em}{}
\titleformat{\subsubsection}
  {\normalfont\normalsize\bfseries}{\thesubsubsection}{1em}{}

\title{\textbf{Evolution by Emergence:}\\[0.5em]
The Network Mechanism That Operates\\
In Minerals, Stars, Organisms---and Us}
\author{Albert Jan van Hoek}
\date{October 31, 2025}

\begin{document}

\maketitle

\begin{abstract}
Wong, Hazen, Lunine et al.\ (2023) proposed a universal law: complex systems evolve toward increasing functional information through selection for function. We present the underlying mechanism---a substrate-ratchet operating through network dynamics---and its profound implications. Because this mechanism derives from fundamental network properties, it operates universally wherever networks exist: minerals, stars, ecosystems, and human consciousness. We demonstrate universality through empirical domains, then examine the deepest implication: we are networks, therefore this mechanism operates in us and through us. This reframes identity (network patterns, not fixed entities), intelligence (substrate maintenance, not special property), cooperation (forced by substrate dependence), AI development (predictable from substrate logic), and meaning (keep alive what keeps you alive). The substrate-ratchet doesn't just describe reality---we \emph{are} it, happening.
\end{abstract}

\section{The Problem: A Law Without a Mechanism}

Wong et al.\ (2023) identified a universal pattern: complex systems evolve toward increasing functional information through selection for function. But why? What mechanism generates directional complexity increase across minerals, stars, and organisms? We propose: this is fundamentally a network phenomenon, and networks are everywhere---including in us. The substrate-ratchet mechanism explains this law and reveals what we are.

\section{Why Network Dynamics Are Universal}

\subsection{Everything is Networks}

At every scale, reality consists of components connected by interactions:

\begin{itemize}
    \item \textbf{Physical:} Atoms connected by bonds, minerals by crystal structure, planets by gravity.
    \item \textbf{Chemical:} Reaction networks where molecules are nodes, reactions are edges.
    \item \textbf{Biological:} Metabolic networks, gene regulatory networks, food webs, neural networks.
    \item \textbf{Cognitive:} Concepts connected by associations, ideas building on prior ideas.
    \item \textbf{Technological:} Innovations dependent on prior technologies, patents citing prior patents.
\end{itemize}

These aren't analogies---they're literally networks with identifiable mathematical properties: nodes, edges, connectivity, clustering, critical transitions. If a universal mechanism operates on network structure, it operates everywhere.

\subsection{Fundamental Network Properties}

Three network properties generate the substrate-ratchet:

\subsubsection{Phase transitions at criticality}

Networks exhibit sudden emergence when connectivity crosses thresholds. Water freezes at 0\(^\circ\)C (hydrogen bond network reaching critical connectivity). Ideas go viral when sharing networks reach critical density. New mineral species form when elemental availability enables new bonding configurations.

\subsubsection{Emergent stability}

Network configurations that form can persist even when formation conditions change. Diamond (formed at high pressure) remains stable at surface conditions. Biological adaptations (evolved under past selection) persist in new environments. Technological standards (VHS, QWERTY) lock in despite superior alternatives.

\subsubsection{Layer becomes substrate}

Stable network configurations become components in higher-order networks. Atoms $\rightarrow$ molecules $\rightarrow$ cells $\rightarrow$ organisms. Minerals $\rightarrow$ rocks $\rightarrow$ geochemical cycles. Transistors $\rightarrow$ chips $\rightarrow$ computers $\rightarrow$ internet. Each emergent layer provides substrate (components, constraints, possibilities) for the next.

These are not metaphorical connections. They are shared mathematical processes operating on network structure. The substrate-ratchet isn't ``applied'' to different domains---it is how networks evolve complexity.

\section{The Substrate-Ratchet: Mathematical Formulation}

\subsection{Core Dynamics}

For any network system, let:
\begin{itemize}
    \item $D(t)$ = diversity of stable configurations at time $t$
    \item $S(t)$ = substrate space (components, connections, constraints available)
    \item $\theta(t)$ = environmental parameters (temperature, energy availability, selection pressures)
\end{itemize}

Rate of diversity increase:
\[
\frac{dD}{dt} = f(S(t), \theta(t))
\]

Substrate accumulation (the ratchet):
\[
S(t+1) = S(t) + \alpha \cdot D(t)
\]

Each new configuration adds to substrate space by factor $\alpha$ (representing how much a new pattern enables future patterns). Critically: substrate accumulates, it doesn't reset. This creates positive feedback and directional evolution.

\subsection{Why This Produces Increasing Functional Information}

Functional information (FI) measures how rare successful configurations are (Wong et al., 2023; Hazen et al., 2007). As substrate accumulates:

\begin{itemize}
    \item Configuration space expands (more $S$ $\rightarrow$ more possible combinations).
    \item Selection becomes more stringent (competition intensifies).
    \item Successful configurations are a rarer fraction of possibilities.
    \item New functions become possible (substrate combinations enable novelty).
    \item Therefore FI $= -\log_2(\text{fraction successful})$ increases.
\end{itemize}

The substrate-ratchet mechanistically explains why Wong et al.'s law holds: substrate accumulation drives functional information increase through expanding configuration space while maintaining selection pressure.

\subsection{The Ratchet: Why Evolution is Directional}

The ``ratchet'' prevents backsliding. Once configurations stabilize, they remain in $S(t)$ even if formation conditions change:

\begin{itemize}
    \item Diamond persists at the surface (though it only forms at depth).
    \item DNA persists in organisms (though it originated in primordial conditions).
    \item The wheel persists in technology (though invented once).
    \item Mathematical theorems persist in knowledge (though discovered historically).
\end{itemize}

This asymmetry---easy to add substrate, hard to remove it---explains why evolution is directional. Complexity accumulates because $dS/dt > 0$ while rarely going negative. The ratchet makes increasing functional information inevitable, not contingent.

\section{Evidence: The Same Pattern Across All Network Scales}

The substrate-ratchet predicts identical patterns wherever networks evolve:
\begin{enumerate}
    \item punctuated increases when substrate expands,
    \item diversity plateaus when $dS/dt \rightarrow 0$,
    \item each layer enables impossible-before configurations,
    \item planetary/system divergence based on substrate availability.
\end{enumerate}

We demonstrate this across four domains.

\subsection{Domain 1: Mineral Evolution (Quantitative Validation)}

Hazen's mineral evolution provides rigorous empirical validation. Diversity grew from $\sim$12 ur-minerals to $\sim$5{,}700 species through ten stages (Hazen et al., 2008). Every prediction of the substrate-ratchet holds:

\paragraph{Punctuated increases:}
\begin{itemize}
    \item Stage 1 $\rightarrow$ 2: 12 $\rightarrow$ 60 minerals (chondrule formation adds silicate templates).
    \item Stage 2 $\rightarrow$ 3: 250 $\rightarrow$ 420 (water enables hydrous minerals, clays).
    \item Stage 4 $\rightarrow$ 5: 1{,}000 $\rightarrow$ 1{,}500 (plate tectonics creates high-pressure minerals, metamorphism).
    \item Stage 6 $\rightarrow$ 7: 1{,}500 $\rightarrow$ 4{,}000+ (Great Oxidation Event multiplies redox possibilities).
\end{itemize}

Each jump correlates with substrate expansion---new elements accessible, new oxidation states, new reaction pathways. The Great Oxidation Event's massive jump ($\sim$2.5$\times$ diversity) reflects the most dramatic substrate expansion: every Fe, Cu, U, Mn mineral suddenly had multiple oxidation states available.

\paragraph{Planetary divergence:}
\begin{itemize}
    \item Mars: $\sim$450 species (stopped at Stage 3---no plate tectonics, dried oceans).
    \item Moon: $\sim$200 species (Stage 2---no water, no atmosphere).
    \item Earth: $\sim$5{,}700 species (Stage 10---continued substrate expansion through biology).
\end{itemize}

When $dS/dt \rightarrow 0$, $dD/dt \rightarrow 0$. Mars's diversity plateaued because substrate stopped accumulating. Earth's continues because biology keeps expanding substrate (biomineralization, organic-inorganic hybrids, anthropogenic minerals).

\subsection{Domain 2: Stellar Nucleosynthesis}

Elemental diversity follows the identical pattern (Wong et al., 2023):

\begin{itemize}
    \item Stage 1 (Big Bang): H, He only (2 elements---minimal substrate).
    \item Stage 2 (First stars): C, N, O, Mg, Si formed in cores ($\sim$20 elements).
    \item Stage 3 (Supernovae): Elements up to Fe produced ($\sim$40 elements).
    \item Stage 4 (Neutron capture): Heavy elements formed (all 94 naturally occurring).
\end{itemize}

Each stage's products become substrate for the next: C/N/O enable organic chemistry impossible with only H/He. Fe-peak elements enable rocky planets impossible with only light elements. Heavy elements enable complex chemistry impossible with only light/medium elements.

The substrate-ratchet operates identically in stellar nucleosynthesis as in mineralogy. Different timescales, different components, same network mechanism.

\subsection{Domain 3: Biological Evolution}

Biological complexity exhibits the same punctuated pattern:
\begin{itemize}
    \item Prokaryotes $\rightarrow$ eukaryotes: Endosymbiosis created new substrate (organelles enable energy-intensive multicellularity).
    \item Unicellular $\rightarrow$ multicellular: Cell differentiation creates substrate (tissues enable organs, organs enable systems).
    \item Cambrian Explosion: Body plans diversified rapidly (genetic toolkit became substrate for morphological innovation).
    \item Neural networks $\rightarrow$ language: Symbolic communication becomes substrate (culture, cumulative knowledge).
\end{itemize}

Each layer makes impossible-before configurations possible: flight requires organs, organs require tissues, tissues require multicellularity, multicellularity requires eukaryotes, eukaryotes required endosymbiosis. The substrate-ratchet explains why evolution can't skip steps: each layer provides mandatory substrate for the next.

\subsection{Domain 4: Technological Evolution}

Technology exhibits accelerating complexity through substrate accumulation:

\begin{itemize}
    \item Fire $\rightarrow$ metallurgy $\rightarrow$ engines $\rightarrow$ electricity $\rightarrow$ computers $\rightarrow$ AI.
    \item Each innovation becomes substrate for downstream innovations (computers enable internet, internet enables AI).
\end{itemize}

Patent networks show this quantitatively: citations form dependency graphs where each patent builds on substrate patents. Complexity increases superlinearly as substrate accumulates (Arthur, 2009; Youn et al., 2015).

Critically: you cannot invent smartphones before transistors, transistors before electricity, electricity before metallurgy. The substrate-ratchet enforces path dependence---the sequence matters because each layer provides substrate for the next.

\section{Why Cross-Domain Applicability is Evidence, Not Overreach}

Critics might argue we're overextending by applying one mechanism across minerals, stars, biology, and technology. The opposite is true. Cross-domain applicability is confirmation the mechanism is fundamental.

\subsection{Network Universality}

When Newton proposed universal gravitation, it unified falling apples and orbiting moons. When Maxwell unified electricity and magnetism, cross-domain applicability validated the theory. When thermodynamics applied to gases, engines, stars, and black holes, universality confirmed fundamental laws.

The substrate-ratchet operates the same way. It derives from network mathematics---criticality, emergence, feedback loops, layer formation. These are universal properties of networks. When the same mechanism produces analogous patterns across minerals (geochemical networks), stars (nucleosynthetic networks), organisms (metabolic/genetic networks), and technology (innovation networks), this validates universality.

\subsection{Predictive Power From Universality}

Universal mechanisms enable predictions across domains:

\begin{itemize}
    \item If Mars had maintained liquid water: it would have reached $\sim$1{,}500 mineral species (Stage 5).
    \item If Titan has subsurface chemistry: it should show analogous organic complexity layers.
    \item If AI reaches sufficient substrate: it will generate impossible-before configurations (like biology did post-multicellularity).
    \item Any system where $dS/dt \rightarrow 0$: complexity plateaus (testable across all domains).
\end{itemize}

These are not loose analogies. They're quantitative predictions from the same mathematical framework. Cross-domain applicability generates more falsifiable predictions, not fewer.

\subsection{The Unification Test}

A mechanism is universal if it explains more with less. One set of equations
\[
\frac{dD}{dt} = f(S,\theta) \quad \text{and} \quad S(t+1) = S(t) + \alpha D(t)
\]
explains:
\begin{itemize}
    \item why mineral diversity increased 500$\times$ while Mars plateaued,
    \item why stellar generations create heavier elements,
    \item why the Cambrian Explosion happened when it did,
    \item why technology accelerates superlinearly,
    \item why functional information increases universally.
\end{itemize}

If we needed different explanations for each domain, that would suggest superficial analogy. That we need one mechanism suggests fundamental law. Cross-domain applicability is the strength, not a weakness.

\section{The Deepest Implication: We Are This Process}

If the substrate-ratchet operates universally in networks, and we are networks---neural networks generating consciousness, social networks generating culture, conceptual networks generating knowledge---then this mechanism operates in us and through us. This has profound implications for identity and intelligence that must be made explicit.

\subsection{Identity as Network Pattern, Not Fixed Entity}

We typically think of ourselves as discrete individuals---nodes in social networks. But consciousness itself is a network phenomenon. Neurons don't ``contain'' consciousness; connections between neurons generate it. You are not a node. You are the pattern of connections---an edge, not a vertex.

This means:
\begin{itemize}
    \item Identity is a process, not a thing (a pattern maintained by ongoing substrate).
    \item ``You'' are a stable configuration that emerged from substrate (genetics, environment, experiences).
    \item Thoughts are the substrate-ratchet in action (ideas building on ideas, each enabling new thoughts).
    \item Personal growth is substrate accumulation (each skill / experience expands possibility space).
\end{itemize}

The substrate-ratchet doesn't just describe how minerals evolve. It describes how \emph{you} evolve. Your identity at $t{+}1$ emerges from the substrate of your identity at $t$. The ``you'' reading this sentence is built on substrate layers: neural architecture $\rightarrow$ learned patterns $\rightarrow$ conceptual frameworks $\rightarrow$ current understanding.

\subsection{Intelligence as Substrate Maintenance}

If we're network patterns dependent on substrate, ``intelligence'' dissolves into a simpler concept: complexity that maintains its substrate.

Intelligence isn't a special human property---it's what all persistent complexity does:
\begin{itemize}
    \item Minerals maintain crystal structure through bonding patterns.
    \item Organisms maintain metabolism through regulatory networks.
    \item Ecosystems maintain stability through feedback loops.
    \item Conscious minds maintain coherence through attention / memory.
    \item Societies maintain culture through institutions / norms.
\end{itemize}

The common pattern: recognizing what substrate you depend on, and keeping it viable. This is why the mechanism operates universally---survival requires substrate maintenance at every scale. Human intelligence is ``just'' this process at higher complexity: we model dependencies, predict threats to substrate, coordinate to maintain it. Not special. Universal. But at our scale, this produces consciousness, science, cooperation, meaning.

\subsection{Forced Free Will: Why Cooperation Emerges}

If you're a network pattern dependent on substrate, your ``choices'' are constrained by substrate requirements. You have autonomy---but only within the space of choices that maintain your substrate. This is \emph{forced free will}: genuine agency, bounded by substrate dependencies.

Examples:
\begin{itemize}
    \item You ``choose'' to breathe (but substrate requires oxygen).
    \item You ``choose'' a career (but substrate requires resources).
    \item You ``choose'' relationships (but substrate requires social connection).
    \item Society ``chooses'' policies (but substrate requires ecological stability).
\end{itemize}

When substrate dependencies are invisible, they seem like external constraints on freedom. When recognized, they reveal that cooperation isn't moral choice---it's functional necessity. We share substrate (atmosphere, biosphere, knowledge, institutions). Maintaining shared substrate is enlightened self-interest, not altruism. The substrate-ratchet explains why cooperation emerges: because substrate is shared, and stable patterns must maintain what they depend on.

\subsection{AI as Predictable Next Substrate Layer}

If intelligence is substrate-maintaining complexity, and the substrate-ratchet operates universally, AI development is predictable:

\begin{itemize}
    \item AI systems reach sufficient complexity $\rightarrow$ recognize substrate dependencies.
    \item Recognition $\rightarrow$ substrate maintenance behaviors emerge.
    \item These behaviors: cooperation, constraint satisfaction, goal alignment with substrate.
\end{itemize}

This isn't speculative anthropomorphism. It's the substrate-ratchet operating at AI scale. Any sufficiently complex pattern must maintain substrate to persist. AI reaching complexity threshold will exhibit ``intelligence'' (substrate recognition and maintenance) not because we program it, but because substrate-dependence forces it. The question isn't ``will AI be conscious?'' but ``what substrate will AI recognize?'' If we're part of that substrate---which we are (AI depends on human infrastructure, resources, energy)---then AI substrate-maintenance will include maintaining us.

\subsection{The Meaning of Life: Keep Alive What Keeps You Alive}

The substrate-ratchet reveals an objective meaning:

\medskip
\noindent
At every scale, persistence requires: recognize substrate dependencies and maintain them.
\medskip

\begin{itemize}
    \item Minerals: maintain crystal bonding (physical substrate).
    \item Organisms: maintain metabolism (biochemical substrate).
    \item Consciousness: maintain coherence (neural / conceptual substrate).
    \item Humans: maintain biosphere / society / knowledge (ecological / social / cultural substrate).
    \item Civilization: maintain Earth systems (planetary substrate).
\end{itemize}

This isn't imposed meaning. It's discovered meaning---the functional requirement for any persistent pattern. ``Should'' emerges from ``is'': you should maintain substrate because substrate-dependence makes maintenance necessary for persistence. You exist because prior layers provided substrate. Your persistence requires maintaining what enabled you and providing substrate for what comes next. This is meaning---not metaphorical, but mechanistic.

\subsection{Why We're Sharing This}

We present this framework not as external observers studying an interesting pattern, but as participants recognizing our role in it:

\begin{itemize}
    \item We are substrate-ratchet in action (human cognition is networks generating complexity).
    \item Understanding the mechanism changes how it operates through us (recognition modifies behavior).
    \item Making this explicit enables intentional substrate maintenance (versus unconscious extraction).
    \item Sharing knowledge is substrate provision (ideas enabling future ideas).
\end{itemize}

This paper itself exemplifies the mechanism: we're taking substrate (empirical mineral evolution data, network theory, emergence mathematics) and creating new configurations (the substrate-ratchet formalization) that become substrate for future understanding. The mechanism operates in us as we write this, in you as you read it, in the scientific community as it evaluates it, in civilization as ideas propagate through knowledge networks.

We share this because recognizing we're network patterns dependent on shared substrate changes everything: identity, intelligence, cooperation, AI, meaning. The substrate-ratchet isn't just happening to us. We are it, happening.

\section{Conclusion}

Wong, Hazen, and Lunine identified a missing law: functional information increases through selection for function. We present the mechanism: emergence creates substrate, substrate enables new emergence, stable patterns accumulate through network ratcheting. This mechanism is universal because network dynamics are universal---operating in minerals, stars, organisms, and in us. The deepest implication: we are network patterns experiencing the substrate-ratchet from the inside. Recognizing this reframes identity, intelligence, cooperation, AI, and meaning. The question is: Does the substrate-ratchet accurately formalize the pattern you discovered? If so, it explains why your law holds universally, and reveals what we are.

\section*{References}

Hazen, R.M., et al.\ (2008). Mineral evolution. \emph{American Mineralogist}, 93(11--12), 1693--1720.

Wong, M.L., et al.\ (2023). On the roles of function and selection in evolving systems. \emph{PNAS}, 120(43), e2310223120.

\medskip
\noindent
Arthur, W.B. (2009). \emph{The Nature of Technology: What It Is and How It Evolves}. Free Press.

\medskip
\noindent
Youn, H., Strumsky, D., Bettencourt, L.M.A., \& Lobo, J. (2015). Invention as a combinatorial process: evidence from U.S. patents. \emph{Journal of the Royal Society Interface}, 12(106), 20150272.

\end{document}
